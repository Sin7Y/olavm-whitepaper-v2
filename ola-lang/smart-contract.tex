\subsection{Smart Contracts}


Ola contracts allow users to write complex business logic that will be deployed to Ola's L2 network, and cross-contract calls can be written between different contracts just like solidity.

\subsubsection{Simple Examples}

The following example shows a recursive and non-recursive Ola smart contract implementation of Fibonacci numbers.

\begin{lstlisting}
contract Fibonacci {

    fn main() {
       fib_non_recursive(10);
    }

    fn fib_recursive(u32 n) -> (u32) {
        if (n == 0 || n == 1) {
            return 1;
        }
        return fib_recursive(n -1) + fib_recursive(n -2);
    }

    fn fib_non_recursive(u32 n) -> (u32) {
        u32 first = 0;
        u32 second = 1;
        u32 third = 1;
        for (u32 i = 2; i <= n; i++) {
             third = first + second;
             first = second;
             second = third;
        }
        return third;
    }

}
\end{lstlisting}

The following shows a simple Person contract that contains a person structure, assigns a value to the Person structure, and reads the status of the person.

\begin{lstlisting}
contract Person {

    enum Sex {
        Man,
        Women
    }

    struct Person {
        Sex s;
        u32 age;
        u256 id;
    }

    Person p;

    fn newPerson(Sex s, u32 age, u256 id) {
        p = Person(s, age, id);
    }

    fn getPersonId() -> (u256) {
        return p.id;
    }

    fn getAge() -> (u32) {
        return p.age;
    }
}
\end{lstlisting}

\subsubsection{Multiple files}

For better project organization and clearer logic, it is common to split the contents of a file into multiple files. Ola language supports the import of another contract within a contract through the `import` keyword.

An example of a multi-file contract is shown below.

\textbf{Contract RectangularCalculator}

\begin{lstlisting}
contract RectangularCalculator {
  
    fn rectangle(u32 w, u32 h) -> (u32 s, u32 p) {
        s = w * h;
        p = 2 * (w + h);
        // Returns a variable with the same name, the return can be ignored
        //return (s, p)
    }
}
\end{lstlisting}

\textbf{Contract ShapeCalculator}

\begin{lstlisting}
contract SquareCalculator {

    fn square(u32 w) -> (u32 s, u32 p) {
        s = w * w;
        p = 4 * w;
        return (s, p);
    }
}
\end{lstlisting}

\textbf{Contract Calculator}

\begin{lstlisting}
import "./RectangularCalculator";
import "./SquareCalculator";

contract Calculator {
  
    fn sum(u32 w, u32 h) -> (u32 s, u32 p) {
        (u32 rectangle_s, u32  rectangle_p) = rectangle(w, h);
        (u32 square_s, u32 square_p) = square(w);
        return (rectangle_s + square_s, rectangle_p + square_p);
    }
}
\end{lstlisting}

