\subsubsection{Ola Compiler Backend}

Detailed diagram of ir->asm flow and structure (expanded below)

Structure:

    lists

    insts

    mcinsts

ABI:

    Function call specification

    Mapping Relationships to Virtual Memory

Devices:

    IR parsing

    Target code generation

        Function call demotion

        Instruction selection

        Register Allocation

        Slot elimination

        Stack frame handling

        Assembly printing

    \begin{itemize}
        \item IR parsing

    IR is composed of the following structure: module -> function -> value -> types.

    Module structure contains Target (triple and datalayout information), Function, Attribute, GlobalVariable.

    Function structure contains name, type information, Visibility, Attribute and Parameter list basic information, and data, Layout.

    Among them, layout contains the sequential logical relationship between basicBlocks and the instructions within them.

    Data contains the specific Value, Instruction, BasicBlock list instances.

    BasicBlock is identified by BasicBlockId and consists of two parts: name and number. Each BB block usually contains one preds and one sucs.

    Instruction is identified by BasicBlockId+InstructionId and usually consists of Opcode, Operand, dest and the Type of its operation.

    Value contains Instruction, Argument, Constant and InlineAsm types.

    Due to the characteristics of the instruction set of olavm, the type is currently mainly i64 type.

        \item IR parsing process

        process: targetDatalayout/targetTriple -> attributeGroup -> localType -> globalVariable -> function -> metadata.

    The function is mainly divided into two parts: parseArgumentList and parseBody.

        \item IR opt pass

    Pre-analysis analysis pass is mainly DominatorTree, conversion transform pass contains dce, mem2reg, sccp.

        \item back-end codegen modules

    bridging ir structure of module, function, and isa related callconv, register and isa, code generation related lower and optimization related pass.

    TargetISA main contains custom TargetInst, Register(RegisterClass/RegisterInfo), lower and modulepasses, callconv, datalayout information.

    Module in addition to inheritance Ir parsed out Module, the description of its Function and ir differ significantly.

    Data information: BasicBlock in the instructions for Target Instruction, register contains VRegs and RegUnit two categories, and contains a vregtoinsts mapping.
    At the same time, Inst in Layout are referred to TargetInst.
    Note that the memory access operations of parameters, local variables, etc. are described by the structure of slot(ptr+offset).

    The lower module provides the process of downgrading LoweringContext to Instruction, and for function calls it also requires copyargstovregs.

    The pass module contains the regalloc and spiller for analyzing the liveness of the pass and the function pass.

        \item back-end process

    Register and insts description

    Function call demotion

    Instruction selection

    eliminateslot

    proepiinserter

    reg allocation and coalescing
    
    \end{itemize}