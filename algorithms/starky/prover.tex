\subsubsection{Generate a Starky Proof}\label{section: starky-generate-proof}

Each STARK table in Starky must generate a zk-STARK proof separately. The proof process for each table is as follows:

\begin{enumerate}
    \item Computational Polynomials and Commitments
        \begin{enumerate}
            \item Generate a Merkle commitment from polynomials of STARK table columns
            \item Divide the instance of the permutation argument into multiple batches, each batch generates a polynomial Z(X)
            \item Calculate polynomials for cross table lookups Z(X)
            \item Generate a Merkle commitment from all Z(X) polynomials above
        \end{enumerate}
    \item Using trace polynomials and Z(X) to generate quotient polynomials: $$quotient(X) = \frac{(\sum C(X)) + (P\_Z(gX)*\prod rhs - P\_Z(X)*\prod lhs) + (CTL\_Z(gX) - CTL\_Z(X)* selector(gX))}{X^n - 1}$$
    \item Divide the quotient polynomials into small polynomials, the degrees of which are quotient\_degree\_factor
    \item Generate a Merkle commitment from quotient polynomials
    \item Calculate the three points of the challenge zeta, g $ \cdot $ zeta and the last point $ g^{n-1} $ in the group H
    \item Open the challenge points on the polynomial, respectively, and get openings
        \begin{enumerate}
        \item Point zeta is opened on
            \begin{itemize}
                \item Trace polynomials
                \item Permutation and cross table lookups polynomials
                \item Quotient polynomials
            \end{itemize}
        \item Point g $ \cdot $ zeta is opened on
            \begin{itemize}
                \item Trace polynomials
                \item Permutation and cross table lookups polynomials
            \end{itemize}
        \item Point $ g^{n-1} $ is opened only on
            \begin{itemize}
                \item Cross table lookups polynomials
            \end{itemize}
        \end{enumerate}
    \item Generate the final polynomial
        \begin{enumerate}
            \item Compress the polynomial opened at each point into one polynomial $$ F_i(X) = \sum_{j} \alpha^j \cdot f_{ij}, i \in [3] $$
            \item Calculate the above $ F_i(X) $ quotient, where $ z \in \{ zeta, g \cdot zeta, g^{n-1} \} $ $$ Q_i(X) = \frac{F_i(X) - F_i(z)}{X - z} $$
            \item Compress all the points to $ Q_i(X) $ get the polynomial of the final FRI $$ Final(X) = \sum_{k=0}^3 \alpha^k * Q_i(X) $$
        \end{enumerate}
    \item Use the final polynomial generated FRI proof fri\_proof, which is opening\_proof
    \item Finally got the proof of a single stark, proof consists of
        \begin{itemize}
            \item trace\_cap
            \item permutation\_ctl\_zs\_cap
            \item quotient\_polys\_cap
            \item openings
            \item opening\_proof
        \end{itemize}
    \item The STARK proofs and public input of all STARK tables are converted into a final proof of the program, called a starky proof
\end{enumerate}

\noindent There are now a total of 3 Merkle commitments

\begin{enumerate}
    \item A Merkle commitment of trace polynomials
    \item A Merkle commitment of permutation\_ctl\_zs polynomials
    \item A Merkle commitment of quotient polynomials
\end{enumerate}

The order of a quotient polynomial is determined by the related constraints and trace degree: quotient\_degree\_factor $ \cdot $ degree . The last point $ g^{n-1} $is for cross-table lookup checking. That is, the product of the last point value of multiple looking tables is equal to the last point value of the looked table.
