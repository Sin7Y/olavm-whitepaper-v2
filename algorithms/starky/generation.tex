\subsubsection{STARK Tables Generation}\label{section: starky-generation-tables}

The program in Ola will generate multiple original STARK tables after execution. There are several tables currently, such as CPU, Memory, Bitwise, RangeCheck, CMP, etc.

Different tables define a different number of columns for storing data such as original data, fixed data, auxiliary data. The values of some columns are determined when the program is executed, and the values of the remaining columns need to be calculated using the program execution results. When the cells of all STARK tables have values, we get the complete STARK tables, and then we can use FFT to interpolate each column in each STARK table into a polynomial. The operation process for each table is as follows:

\begin{enumerate}
    \item Iterate each row of table, calculating the values of all instruction-related columns
    \item Calculate the value of the permutation argument related columns
    \item Calculate the value of the lookup related columns
    \item Interpolate all columns into polynomials using FFT
    \item Returns a column polynomial array and public input
\end{enumerate}
