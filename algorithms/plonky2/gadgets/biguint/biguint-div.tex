\paragraph{biguint-div}

Note that div-rem has the same constraints logic with div

\subparagraph{Target}
Implement the division of two biguints.

\subparagraph{Constraints logic}
\begin{itemize}
    \item Not implement div-algrithem directly;
    \item Use nondeterministic feature to check div-logic;
    \item Check \verb|div * b + rem = a|;
    \item Check \verb|rem < b|.
\end{itemize}

\subparagraph{Process layout}
See \figref{fig:biguint-div-layout}.
\begin{figure}[!ht]
    \centering
    \includegraphics[width=0.6\textwidth]{biguint-div-layout.jpg}
    \caption{biguint-div layout}
    \label{fig:biguint-div-layout}
\end{figure}

\subparagraph{Constraints info and costs}
\begin{itemize}
    \item Gate type num: 5 (U32ArithmeticGate, U32AddManyGate (num-addends: 3), U32AddManyGate (num-addends: 4), ComparisionGate, ArithmeticGate)
    \item Gate instance num: $3 + 3 + 4 + 3 = 13$
    \item U32ArithmeticGate num: 3
    \item U32AddManyGate num: 3
    \item ComparisionGate num: 4
    \item ArithmeticGate num: 3
    \item copy-constraints: $3 \times 8 + 4 + 5 + 4 + 4 \times 6 + 7 + 1 + 26 + 5 = 100$
    \item max-degree: 4
\end{itemize}
