\subsubsection{Circuit config} \label{sec:circuit-config}

There are some types of configure in plonky2, let us make some clarification on this first:
\begin{itemize}
    \item \verb|num_wires| is the maximum column number one row;
    \item \verb|routed_num_wires| is the maximum wires one row;
    \item \verb|num_constants| is the maximum constants one row; 
    \item \verb|use_base_arithmeric_gate| if the flag of base gate or extension gate;
    \item \verb|security_bit| is the security level of prove system;
    \item \verb|num_challenges| is the security parameter to get security\_bit level;
    \item \verb|zero_knowledge| is the flag of blind;
    \item \verb|max_quotient_degree_factor| is the order parameter of constraint; 
    \item \verb|rate_bits| is the extension parameters for domain extension
    \item \verb|cap_height| is the special parameter of Merkle tree;
    \item \verb|proof_of_work_bit| is the hardness parameter of query points;
    \item \verb|reduction_strategy| id the parameters of Fri-reduction;
\end{itemize}  

\paragraph{standard-recursion-config}

\hspace*{\fill} \\
\begin{lstlisting}[language=rust]
pub fn standard_recursion_config() -> Self {
    Self {
        num_wires: 135,
        num_routed_wires: 80,
        num_constants: 2,
        use_base_arithmetic_gate: true,
        security_bits: 100,
        num_challenges: 2,
        zero_knowledge: {\color{green}false},
        max_quotient_degree_factor: 8,
        fri_config: FriConfig {
            rate_bits: 3,
            cap_height: 4,
            proof_of_work_bits: 16,
            reduction_strategy: FriReductionStrategy::ConstantArityBits(4, 5),
            num_query_rounds: 28,
        },
    }
}  
\end{lstlisting}

\paragraph{standard-recursion-zk-config}

\hspace*{\fill} \\
\begin{lstlisting}[language=rust]
pub fn standard-recursion-zk-config() -> Self {
    Self {
        num_wires: 135,
        num_routed_wires: 80,
        num_constants: 2,
        use_base_arithmetic_gate: true,
        security_bits: 100,
        num_challenges: 2,
        zero_knowledge: {\color{green}true},
        max_quotient_degree_factor: 8,
        fri_config: FriConfig {
            rate_bits: 3,
            cap_height: 4,
            proof_of_work_bits: 16,
            reduction_strategy: FriReductionStrategy::ConstantArityBits(4, 5),
            num_query_rounds: 28,
        },
    }
}
\end{lstlisting}
  
\subsubsubsection{standard-ecc-config}

\hspace*{\fill} \\
\begin{lstlisting}[language=rust]
pub fn standard_ecc_config() -> Self {
    Self {
        num_wires: 136,
        num_routed_wires: 80,
        num_constants: 2,
        use_base_arithmetic_gate: true,
        security_bits: 100,
        num_challenges: 2,
        zero_knowledge: false,
        max_quotient_degree_factor: 8,
        fri_config: FriConfig {
            rate_bits: 3,
            cap_height: 4,
            proof_of_work_bits: 16,
            reduction_strategy: FriReductionStrategy::ConstantArityBits(4, 5),
            num_query_rounds: 28,
        },
    }
}   
\end{lstlisting}
  
\paragraph{wide-ecc-config}


\begin{lstlisting}[language=rust]
pub fn wide_ecc_config() -> Self {
    Self {
        num_wires: 234,
        num_routed_wires: 80,
        num_constants: 2,
        use_base_arithmetic_gate: true,
        security_bits: 100,
        num_challenges: 2,
        zero_knowledge: false,
        max_quotient_degree_factor: 8,
        fri_config: FriConfig {
            rate_bits: 3,
            cap_height: 4,
            proof_of_work_bits: 16,
            reduction_strategy: FriReductionStrategy::ConstantArityBits(4, 5),
            num_query_rounds: 28,
        },
    }
} 
\end{lstlisting}
 
