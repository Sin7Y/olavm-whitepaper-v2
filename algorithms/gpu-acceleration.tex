\subsection{GPU Acceleration}\label{section: gpu-acceleration}
The polynomial computations of ZK mainly consists of multiple NTTs and INTTs. In our ZK framework, the maximum size of NTTs is $2^{23}$. The large-size NTTs result in sigificant challenges for both off-chip memory accesses and on-chip compute resources, due to the irregular strided access patterns similar to classical FFTs. The GPU acceleration section takes full advantage of GPU multi-threaded parallel processing to speed up NTT calculations.
\subsubsection{NTT Computations}
The NTT computation is defined as
\[\hat{a} = NTT{(a)}\]
Where $a$ and  $\hat{a}$ are two $N$ size arrays, and  $\hat{a}[i] = \sum_{j=0}^{N-1} a[j] \omega_N^{ij}$. Here $a[i]$ and $\hat{a}[i]$ are $\lambda$-bit scalars in a finite field. And $\omega_N$is the $N$-th root of unity in the same field, also is called twiddle factor, which is a constant value for a specific size of $N$.

Similar to classical FFTs, optimized and improved performance implementations of the NTT are of necessity and fundamental essentials. Cooley-Tukey processing is a commonly used NTT algorithm in science and engineering. For example, the Cooly-Tukey algorithm reduces the computational complexity from $O(N^2)$ to $O(NlogN)$. Its fundamental idea is to recursively decompose a large-size NTT into several linear combinations of smaller NTTs. The butterfly network of Cooley-Tukey NTT is a decimation-in-time network whereby its twiddle factor appears on the input side of calculation and input data are arranged in bit-reversed order. Fig.1 illustrates the butterfly networks of Cooley-Tukey NTT. The NTT optimization mainly involves butterfly network and butterfly computing.
\subsubsection{2-D NTT}
NTT is an important kernel commonly used in our ZK scheme. In fact, there exist many accelerator designs for NTT. However, the polynomial computations in our ZK scheme have substantially larger scales than those addressed in other ZK. It requires multiple NTTs of up to $2^{23}$ elements, Such large sizes can hardly be satisfied by any previous GPU design.

To overcome the large-sizes challenges, we adopt a 2-D NTT algorithm to decompose a large NTT into multiple smaller NTT kernels (4096/8192). This allows GPU to only implement smaller NTT modules, which can fit into the compute resources (e.g. the stream processor number of on GPU). Then the original large-size NTT can be calculated by iteratively using the smaller NTT.

The overview of 2-D NTT algorithm is shown in Figure 2. First of all, a large NTT size N is decomposed into P rows and Q columns, with N=P×Q. Then the N-size NTT can be calculated by several P-size and Q-size smaller NTTs. Firstly, the P-size NTT is parallel computing using the same twiddle factors for each of the Q columns. The output is still a P-size vector. Before the second dimensional NTT, a corresponding twiddle factor $\omega^{ij}$ is multiplied to each element of the output vector. Finally, a Q-size NTT for each of the P rows is parallel computing. The output column-major order elements is the result of the large-size NTT.
\subsubsection{GPU & CUDA}
