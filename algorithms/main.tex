\section{Algorithms} \label{sec:algorithms}
Ola uses zero-knowledge proof technologies to ensure the correct execution of the program and protect privacy. Zero-knowledge proof technologies enable one party to prove to another party that a statement is true without revealing any additional information. 

Two of the most compelling zero-knowledge technologies on the market today are zk-STARK and zk-SNARK. Zk-SNARK stands for Zero-Knowledge Succinct Non-interactive Argument of Knowledge. They were introduced in a \href{https://dl.acm.org/doi/10.1145/2090236.2090263}{2012 paper} co-authored by Nir Bitansky, Ran Canetti, Alessandro Chiesa, and Eran Tromer. Zk-STARK stands for Zero-Knowledge Scalable Transparent Argument of Knowledge and is a zero-knowledge proof system that was introduced as an alternative to SNARKs in a \href{https://starkware.co/wp-content/uploads/2022/05/STARK-paper.pdf}{2018 paper} by Eli Ben-Sasson, Iddo Bentov, Yinon Horesh, and Michael Riabzev. We have a table depicting some of the high-level differences between the two technologies.
\subsection{Starky for ZKVM}\label{section: starky-zkvm}
\subsection{Plonky2} \label{sec:Plonky2}

plonky2 = plonkish contraints system + Deep-Fri commitment + Goldilocks field. We are plan to use plonky2 as our recursive layer2 and layer3(supported in the future) because it's more suitable
for specific computation than starky which is more suitable for ZKVM. Should be noted that the designs described in this is the original design of plonky2, will be more convenient to learn for
reader.

\subsubsection{Circuit config} \label{sec:circuit-config}

\input{algorithms/plonky2/circuit-config/standard-recursion-config}
\paragraph{standard-recursion-zk-config}

\hspace*{\fill} \\
\begin{lstlisting}[language=rust]
pub fn standard-recursion-zk-config() -> Self {
    Self {
        num_wires: 135,
        num_routed_wires: 80,
        num_constants: 2,
        use_base_arithmetic_gate: true,
        security_bits: 100,
        num_challenges: 2,
        zero_knowledge: true,
        max_quotient_degree_factor: 8,
        fri_config: FriConfig {
            rate_bits: 3,
            cap_height: 4,
            proof_of_work_bits: 16,
            reduction_strategy: FriReductionStrategy::ConstantArityBits(4, 5),
            num_query_rounds: 28,
        },
    }
}
\end{lstlisting}
  
\input{algorithms/plonky2/circuit-config/standard-ecc-config}  
\input{algorithms/plonky2/circuit-config/wide-ecc-config} 

\subsubsection{Gates} \label{sec:gates}

Each gate is a row with 135 columns. As different custom gate has different complexity, for some complex gate 135 columns may only constraints one
operation while for other simple gate 135 columns may constraints several operations. The index of operation in a row is called slot.

Steps to use a custom gate:
\begin{itemize}
    \item Ensure the type of gate;
    \item Ensure the row and slot location of gate;
    \item Ensure the wires layout of gate;
    \item Ensure the consistency between public data and wires of gate;
    \item Ensure the row location of public data;\
    \item Finilize the trace table;
    \item Constraint for trace table;
    \item Generate proof;
\end{itemize}

For the convenience of description, the trace tables in this paper only show one operation which is slot is 0.

\paragraph{Custom gates}
\paragraph{arithmetic\_base}

\hspace*{\fill}

\indent ArithmeticGate is a gate that can perform a multiply-add with constants, i.e.
\[ \text{res} = \text{cons\_0} \times \text{mul\_0} \times \text{mul\_1} + \text{cons\_1} \times \text{add}. \]

The structure of the gate is shown in \figref{fig:arithmetic-gate}.

\begin{figure}[!ht]
    \centering
    \includegraphics[width=0.6\textwidth]{gates/arithmetic_base.jpeg}
    \caption{ArithmeticGate}
    \label{fig:arithmetic-gate}
\end{figure}

There's only one constraint per operation, and the degree is 3.

\paragraph{arithmetic\_extension}

To understand the design principle of this Gate, we must first understand \href{https://en.wikipedia.org/wiki/Field_extension#Extension_field}{Field extension}. 


Taking Plonky2's GoldilocksField as an example, we give the extension field elements under quadratic, quartic and quintic extensions respectively in \figref{fig:goldilocksfield-extension}.

\begin{figure}[!ht]
    \centering
    \includegraphics[width=0.6\textwidth]{gates/arthmetic_extension_ext.jpeg}
    \caption{GoldilocksField Extension}
    \label{fig:goldilocksfield-extension}
\end{figure}

It is easy to see that for QuadraticExtension Field, the elements on its domain take the form $a+b\sqrt{7}, \ a,b \in F_p$.
It can be seen that on the quadratic extension domain, there are $p^2$ elements and the original domain is a subset of the quadratic extension domain.

ArithmeticExtensionGate is also a gate which can perform a weighted multiply-add, i.e.
\[ \text{res} = \text{cons\_0} \times \text{mul\_0} \times \text{mul\_1} + \text{cons\_1} \times \text{add} \]
The elements of the QuadraticExtension Field are represented in the form $[a, b]$, so the Gate design for arithmetic\_extension has the following form:

The structure of gate is shown in \figref{fig:arthmetic-extension}.
\begin{figure}[!ht]
    \centering
    \includegraphics[width=0.6\textwidth]{gates/arthmetic_extension.jpeg}
    \caption{ArithmeticExtensionGate}
    \label{fig:arthmetic-extension}
\end{figure}

There's only one constraint per operation, and degree is 3.

\paragraph{base\_sum}

\hspace*{\fill}

\indent BaseSumGate is used to constrain the input to be composed of limbs that are arranged in little-endian. There are two kinds of constraints:

For each limb, limb is in range $[0, \text{base})$:
\[ \sum_{i=0}^{\text{base}}(\text{limb}_i - i) = 0. \]

Input is composed of limbs:
\[ \text{input} = \sum_{i=0}^{n-1} \text{limb}_{n-1-i} \times \text{base}^i. \]

The structure of the gate is shown in \figref{fig:base-sum}.
\begin{figure}[!ht]
    \centering
    \includegraphics[width=0.6\textwidth]{gates/base_sum.jpeg}
    \caption{BaseSumGate}
    \label{fig:base-sum}
\end{figure}

There's 1 constraint for the sum check and 8 constraints for the limbs' range check. The degree of the gate is $2^{\text{limb\_size}}$.

\paragraph{exponentiation}

ExponentiationGate is a gate for raising a value to a power. The trace table contains the base, bits of the exponent, output, and intermediate value of the bits.

Take $A^{21} = A^{10101_b}$ for example to describe intermediate value, the bits are [1, 0, 1, 0, 1].

\begin{enumerate}
    \item Current bit = 1, we start from 1, and times $A^{bit}$ we get $A$
    \item Current bit = 0,
    \begin{itemize}
        \item Square prev\_intermediate\_value $A^{1_b << 1} = A^{10_b}$
        \item Then times $A^{bit}$ we get $A^{10_b} \times A^0 = A^{10_b}$
    \end{itemize}
    \item Current bit = 1,
    \begin{itemize}
        \item Square prev\_intermediate\_value $A^{10_b << 1} = A^{100_b}$
        \item Then times $A^{bit}$ we get $A^{100_b} \times A = A^{101_b}$
    \end{itemize}
    \item Current bit = 0,
    \begin{itemize}
        \item Square prev\_intermediate\_value $A^{101_b << 1} = A^{1010_b}$
        \item Then times $A^{bit}$ we get $A^{1010_b} \times 1 = A^{1010_b}$
    \end{itemize}
    \item Current bit = 1,
    \begin{itemize}
        \item Square prev\_intermediate\_value $A^{1010_b << 1} = A^{10100_b}$
        \item Then times $A^{bit}$ we get $A^{10100_b} \times A = A^{10101_b}$
    \end{itemize}
\end{enumerate}

And we get the last intermediate value $A^{10101_b}$ which should be equal to the output.

Let's take another example of a specific number $2^{13} = 2^{1101_b}$, and have a look at the trace cell:
\begin{center}
    \begin{tabular}{ |c|c|c|c|c|c|c|c|c|c| }
        \hline
        base & b\_0 & b\_1 & b\_2 & b\_3 & output & inter\_0 & inter\_1 & inter\_2 & inter\_3 \\
        \hline
        2 & 1 & 0 & 1 & 1 & 8192 & 2 & 8 & 64 & 8192 \\
        \hline
    \end{tabular}
\end{center}

The structure of the gate is shown in \figref{fig:exponetiation-gate}.
\begin{figure}[!ht]
    \centering
    \includegraphics[width=0.6\textwidth]{gates/exponentiation.jpeg}
    \caption{ExponentiationGate}
    \label{fig:exponetiation-gate}
\end{figure}

Each step result is constrainted with intermediate values, and output is constrained with the final intermediated value, a total of $bits + 1$ constraints.
The degree of the gate is 4, which is determined by the intermediate calculation:
\begin{lstlisting}[language=rust]
let computed_intermediate_value =
            prev_intermediate_value * (cur_bit * base + not_cur_bit);
\end{lstlisting}

\subsubsubsection{Poseidon}

\hspace*{\fill}

\indent \href{https://www.poseidon-hash.info/}{Poseidon} is a hash function designed for the Zero-Knowledge proof system.
Its calculation process is rough as follows:

\begin{figure}[!ht]
    \centering
    \includegraphics[width=0.5\textwidth]{gates/poseidon_process.jpeg}
    \caption{Construction of poseidon}
    \label{fig:poseidon-process}
\end{figure}

Each round function of Poseidon permutation consists of the following three components.
\begin{enumerate}
    \item ARC(.): AddRoundConstants
    \item S: SubWords
    \item M(.): MixLayer
\end{enumerate}

The Trace of the gate is like this:
\begin{figure}[!ht]
    \centering
    \includegraphics[width=0.6\textwidth]{gates/poseidon.jpeg}
    \caption{PoseidonGate}
    \label{fig:poseidon-gate}
\end{figure}

\begin{itemize}
    \item input: components of the input, 12 elements.
    \item output: components of the output, 12 elements.
    \item swap: 0 or 1, Indicates whether the first four elements of the input are swapped with the last four elements.
    \item delta: used when swap is 1, $\text{delta}_i = \text{swap} \times (\text{input}_{\text{rhs}} - \text{input}_{\text{lhs}})$.
    \item green region: full rounds, ri\_1~ri\_11 is the state of each round.
    \item yellow region: partial rounds, each element is state[0] of each round.
\end{itemize}

Calculation process and related constraints:
\begin{itemize}
    \item Assert that swap is binary. (1 constraint)
    \begin{lstlisting}[language=rust]
constraints.push(swap * (swap - F::Extension::ONE))
    \end{lstlisting}
    \item Assert $\text{delta}_i = \text{swap} \times (\text{rhs} - \text{lhs})$. (4 constraints)
    \begin{lstlisting}[language=rust]
for i in 0..4 {
    ....
    constraints.push(swap * (input_rhs - input_lhs) - delta_i);
}
    \end{lstlisting}
    \item Initialize state: when swap=0, state=input; when swap=1, state is swapped input:
    \begin{figure}[!ht]
        \centering
        \includegraphics[width=0.6\textwidth]{gates/poseidon_state_init.jpeg}
        \caption{Poseidon State Init}
        \label{fig:poseidon-state-init}
    \end{figure}
    \begin{lstlisting}[language=rust]
for i in 0..4 {
    ...
    state[i] = vars.local_wires[input_lhs] + delta_i;
    state[i + 4] = vars.local_wires[input_rhs] - delta_i;
}
for i in 8..SPONGE_WIDTH {
    state[i] = vars.local_wires[Self::wire_input(i)];
}
    \end{lstlisting}
    \item Begin first full rounds calculation, for each round r (which is 0--3):
    \begin{itemize}
        \item Perform ARC: Add each element of state to the pre-generated value at a particular position in the array.
        \begin{lstlisting}[language=rust]
for i in 0..WIDTH {
    state[i] += F::from_canonical_u64(ALL_ROUND_CONSTANTS[i + WIDTH * round_ctr]);
}
        \end{lstlisting}
        \item Except for r=0, constrain each element of the state calculated in the previous round (the green part of the first slice of the figure). 
        (12 constraints per round, a total of 36 constraints)
        \item Perform SubWords: Turn state element by element x into $x \mapsto x^7$
        \item Perform MixLayer: Each element of the state is updated according to itself and a pre-generated array.
        \begin{lstlisting}[language=rust]
// r is the index of state elements here.
let mut res = F::ZERO;
    for i in 0..WIDTH {
    res += v[(i + r) % WIDTH] * F::from_canonical_u64(Self::MDS_MATRIX_CIRC[i]);
}
res += v[r] * F::from_canonical_u64(Self::MDS_MATRIX_DIAG[r]);
        \end{lstlisting}
    \end{itemize}
    \item Perform partial rounds:
    \begin{itemize}
        \item Perform ARC
        \begin{lstlisting}[language=rust]
for i in 0..12 {
    if i < WIDTH {
        state[i] += F::from_canonical_u64(Self::FAST_PARTIAL_FIRST_ROUND_CONSTANT[i]);
    }
}
        \end{lstlisting}
        \item Processing of state with $11 \times 11$ MDS (maximum distance separable) matrix.
        \begin{lstlisting}[language=rust]
result[0] = state[0];
for r in 1..12 {
    if r < WIDTH {
        for c in 1..12 {
            if c < WIDTH {
                let t = F::from_canonical_u64(
                    Self::FAST_PARTIAL_ROUND_INITIAL_MATRIX[r - 1][c - 1],
                );
                result[c] += state[r] * t;
            }
        }
    }
}
result
        \end{lstlisting}
        \item Perform 22 round sbox, for the first 21 rounds (r = 0--21):
        \begin{itemize}
            \item Take sbox\_in(yellow elements in the figure), and constrains state[0]=sbox\_in -- 21 rounds totally 21 constraints.
            \item \verb|state[0] = state[0]^7|
            \item \verb|state[0] += FAST_PARTIAL_ROUND_CONSTANTS[r]|
            \item Perform mds to state.
        \end{itemize}
        \item For the 22th round:
        \begin{itemize}
            \item \verb|state[0] = sbox_in| (i constraint)
            \item \verb|state[0] = state[0]^7|
            \item Perform mds to state.
        \end{itemize}
    \end{itemize}
    \item Perform second round "Full rounds", same with the first round. (the green part of the second slice of the figure). (12 constraints, 4 rounds totally of 48 constraints)
    \item Asserts computed result equals output. (12 constraints)
    \begin{lstlisting}[language=rust]
for i in 0..SPONGE_WIDTH {
    constraints.push(state[i] - vars.local_wires[Self::wire_output(i)]);
}
    \end{lstlisting}
\end{itemize}

The constraints of this gate in total is 123, the degree is 7 (when performing s-box, making $\text{state}[i] \mapsto \text{state}[i]^7$).

\paragraph{poseidon\_mds}

This gate is used for constrain outputs of poseidon mds.

\begin{figure}[!ht]
    \centering
    \includegraphics[width=0.6\textwidth]{gates/poseidon_mds.jpeg}
    \caption{PoseidonMdsGate}
    \label{fig:poseidon-mds}
\end{figure}

computed\_output is calculated from input, and constrained with output element by element, total 12 constraints (degree 1).
\begin{lstlisting}[language=rust]
let inputs: [_; SPONGE_WIDTH] = (0..SPONGE_WIDTH)
    .map(|i| vars.get_local_ext_algebra(Self::wires_input(i)))
    .collect::<Vec<_>>()
    .try_into()
    .unwrap();
let computed_outputs = Self::mds_layer_algebra(&inputs);
(0..SPONGE_WIDTH)
    .map(|i| vars.get_local_ext_algebra(Self::wires_output(i)))
    .zip(computed_outputs)
    .flat_map(|(out, computed_out)| (out - computed_out).to_basefield_array())
    .collect()
\end{lstlisting}

\paragraph{random\_access}

RandomAccessGate is used for verify that an element matches a value in the list.

\begin{figure}[!ht]
    \centering
    \includegraphics[width=0.8\textwidth]{gates/random_access.jpeg}
    \caption{RandomAccessGate}
    \label{fig:random-access}
\end{figure}

\begin{itemize}
    \item item: list items.
    \item copy: index of target element in the list
    \item claimed: target element
    \item bit\_i: bits for the i-th copy
\end{itemize}

For each copy:
\begin{itemize}
    \item Constrain bits are 0 or 1. -- bits constraints for each copy, A total of num\_copied*bits constraints.
    \begin{lstlisting}[language=rust]
for &b in &bits {
    constraints.push(builder.mul_sub_extension(b, b, b));
}
    \end{lstlisting}
    \item Constraint copy consists of bits. -- 1 constraint for each copy, A total of num\_copied constraints.
    \begin{lstlisting}[language=rust]
let reconstructed_index = bits
    .iter()
    .rev()
    .fold(zero, |acc, &b| builder.mul_add_extension(acc, two, b));
constraints.push(builder.sub_extension(reconstructed_index, access_index));
    \end{lstlisting}
    \item For each bits, reconstruct items with 2-elements-tuple, select first element when bit is 0 and select second when bit is 1.
    After the bits round, only one element remains, that is, the index element corresponding to bits, constraint it with claimed.
    \begin{figure}[!ht]
        \centering
        \includegraphics[width=0.8\textwidth]{gates/random_access_example.jpeg}
        \caption{Random Access Example}
        \label{fig:random-access-example}
    \end{figure}
    \begin{lstlisting}[language=rust]
for b in bits {
    list_items = list_items
        .iter()
        .tuples()
        .map(|(&x, &y)| builder.select_ext_generalized(b, y, x))
        .collect()
}
// Check that the one remaining element after the folding is the claimed element.
debug_assert_eq!(list_items.len(), 1);
constraints.push(builder.sub_extension(list_items[0], claimed_element));
    \end{lstlisting}
\end{itemize}

Finally, the constant is constrained -- A total of num\_extra\_constraints constraints.

In summary, there're $\text{num\_copies} \times (\text{bits} + 2) + \text{num\_extra\_constants}$ constraints. Degree is $\text{bits} + 1$ which happens when repeatedly folding the list. 

\paragraph{reducing}

ReducingGate is used for computes $\text{output} = \text{old\_acc} + \sum C_i*\alpha^i$ in base field.

Trace structure for this gate is like:

\begin{figure}[!ht]
    \centering
    \includegraphics[width=0.5\textwidth]{gates/reducing.jpeg}
    \caption{ReducingGate}
    \label{fig:reducing}
\end{figure}

The constraint flow is relatively intuitive, initializing acc to old\_acc, then cumulative computation of polynomials by coeff in turn, 
and constraining the intermediate results of each step with acc.

\begin{lstlisting}[language=rust]
for i in 0..self.num_coeffs {
    let coeff = builder.convert_to_ext_algebra(coeffs[i]);
    let mut tmp = builder.mul_add_ext_algebra(acc, alpha, coeff);
    tmp = builder.sub_ext_algebra(tmp, accs[i]);
    constraints.push(tmp);
    acc = accs[i];
}
\end{lstlisting}

The number of constraints is equal to the number of coefficients. Polynomial degree is 2 which happens when calculating $\text{coeff}_i * \alpha$, sum up does not increase degree.

ReducingExtensionGate is like ReducingGate, just computations happen in extension field, constrain logic is all the same.

\paragraph{high\_degree\_interpolation}

InterpolationGate is used for interpolation a polynomial, whose points are a (base field) coset of the multiplicative subgroup 
with the given size, and whose values are extension field elements. As for HighDegreeInterpolationGate,  allows constraints of variable degree, 
up to \verb|1 << subgroup_bits|. The higher degree is a tradeoff for less gates (than LowDegreeInterpolationGate).


HighDegreeInterpolationGate trace is shown in \figref{fig:high-degree-interpolation}.

\begin{figure}[!ht]
    \centering
    \includegraphics[width=0.5\textwidth]{gates/high_degree_interpolation.jpeg}
    \caption{HighDegreeInterpolationGate}
    \label{fig:high-degree-interpolation}
\end{figure}


Constraints:
\begin{itemize}
    \item Bring each point(from the point-value pairs) into the coefficient polynomial to compute the computed\_value, 
    and compare the constraint with the value(from the point-value pairs). -- A total of $2^{\text{subgroup\_bits}}$ constraints.
    \begin{lstlisting}
for (i, point) in coset.into_iter().enumerate() {
    let value = vars.get_local_ext_algebra(self.wires_value(i));
    let computed_value = interpolant.eval_base(point);
    constraints.extend((value - computed_value).to_basefield_array());
}
    \end{lstlisting}
    coset: $[sg, sg^2,...,sg^{2^{\text{subgroup\_bits}}}], \ s=\text{shift}$
    \item Evaluate the coefficient-form polynomial at evaluation point, and constrain with ev. -- 1 constraint.
\end{itemize}

The degree of this gate equals the number of points (num\_points): max point power is $\text{num\_points} - 1$, and multiplication by coefficient adds 1 degree.

Number of constraints equals $\text{num\_points} + 1$: num\_points for consistency between the coefficients and the point-value pairs, 1 constraints for the evaluation value. 

\paragraph{low\_degree\_interpolation}

InterpolationGate is used for the interpolation of a polynomial, whose points are a (base field) coset of the multiplicative subgroup 
with the given size, and whose values are extension field elements. As for LowDegreeInterpolationGate,  all constraints are degree <= 2, 
low degree is a tradeoff for more gates(than HighDegreeInterpolationGate).

LowDegreeInterpolationGate trace is shown in \figref{fig:low-degree-interpolation}.

\begin{figure}[!ht]
    \centering
    \includegraphics[width=0.6\textwidth]{gates/low_degree_interpolation.jpeg}
    \caption{LowDegreeInterpolationGate}
    \label{fig:low-degree-interpolation}
\end{figure}

Constraints:
\begin{itemize}
    \item Constrain powers of shift, from $\text{shift}^2$ to $\text{shift}^{n-1}$, a total of $2^{\text{subgroup\_bits}}-2$ constraints.
    \begin{lstlisting}[language=rust]
for i in 1..self.num_points() - 1 {
    constraints.push(powers_shift[i - 1] * shift - powers_shift[i]);
}
    \end{lstlisting}
    \item Bring each point(from the point-value pairs) into the coefficient polynomial to compute the computed\_value 
    and compare the constraint with the value(from the point-value pairs). -- A total of $2^{\text{subgroup\_bits}}$ constraints.
    \item Constrain powers of evaluation point. -- A total of $2^{\text{subgroup\_bits}}-2$ constraints.
    \item Evaluate the coefficient-form polynomial at the evaluation point and constrain it. -- 1 constraint.
\end{itemize}

As can be seen from the above constraint description, the number of constraints is $3*2^{\text{subgroup\_bits}}-3$, degree of LowDegreeInterpolationGate is 2.

\paragraph{U32 gates}
\subsubsubsection{add\_many\_u32}

\hspace*{\fill}

\indent U32AddManyGate is a gate to perform addition on num\_addends different 32-bit values, plus a small carry. 
There can be up to 16 operations per gate.

The gate structure is like \figref{fig:add-many-u32}.

\begin{figure}[!ht]
    \centering
    \includegraphics[width=0.6\textwidth]{gates/add_many_u32.jpeg}
    \caption{U32AddManyGate}
    \label{fig:add-many-u32}
\end{figure}

Constraints for each operation:
\begin{itemize}
    \item Constrain the result of addends summation equals results of res and carry\_out calculation. -- 1 constraint with degree 1
    \begin{lstlisting}[language=rust]
let base = F::Extension::from_canonical_u64(1 << 32u64);
let combined_output = output_carry * base + output_result;
constraints.push(combined_output - computed_output);
    \end{lstlisting}
    \item Limbs range check. -- 18(limbs) constraints with degree 4.(limbs are all 2-bits)
    \begin{lstlisting}[language=rust]
let product = (0..max_limb)
    .map(|x| this_limb - F::Extension::from_canonical_usize(x))
    .product();
constraints.push(product);
    \end{lstlisting}
    \item Constrain limbs for res and carry\_out. -- 2 constraints with degree 1.
\end{itemize}

In summary, there are 21 constraints for each operation. The degree of the gate is 4 which is needed by the 4-bits limbs range check.

\paragraph{arithmetic\_u32}

U32ArithmeticGate gate is used for compute $\text{res} = \text{mul\_0} \times \text{mul\_1} + \text{add}$. Res is store in res\_lo and res\_hi each of which can be represented by 15 limbs.

Gate structure is like \figref{fig:arithmetic-u32}.

\begin{figure}[!ht]
    \centering
    \includegraphics[width=0.6\textwidth]{gates/arithmetic_u32.jpeg}
    \caption{U32ArithmeticGate}
    \label{fig:arithmetic-u32}
\end{figure}

Constraints for each operation:
\begin{itemize}
    \item Constrain res is not overflow (less equal than \verb|max_u32 * max_u32 + max_u32|). -- 1 constraint with degree 2.
    \item Constrain combined((by res\_lo and res\_hi)) output equals computed (by \verb|mul_0 * mul_1 + add|) output. -- 1 constraint with degree 2.
    \item Limbs range check. -- 32 (limbs) constraints with degree 4. (limbs are all 2-bits)
    \item Constrain limbs for res\_lo and res\_hi. -- 2 constraints with degree 1.
\end{itemize}

In summary, there are 36 constraints for each operation. Degree of the gate is 4 which is needed by 4-bits limbs range check.

\paragraph{comparison}

ComparisonGate is a gate for checking that one value is less than or equal to another. 
Compared numbers are divided into chunks, and chunk size and the number are configurable.

For a ComparisonGate with 8 4-bits chunks is like \figref{fig:comparison}.

\begin{figure}[!ht]
    \centering
    \includegraphics[width=0.6\textwidth]{gates/comparison.jpeg}
    \caption{ComparisonGate}
    \label{fig:comparison}
\end{figure}

The main idea of constraints:
\begin{itemize}
    \item Consistency of the sliced chunks and the original values.
    \item Range check for each chunk.
    \item If the chunks are equal, the difference is 0 and there is no inverse.
    \item Chunk by chunk so for most significant diff equals related intermediate\_value.
    \item If first <= second, the top \verb|n+1|-th bit of $(2^n + \text{most\_significant\_diff})$ will be 1.
\end{itemize}

\paragraph{range\_check\_u32}

U32RangeCheckGate is a gate which can decompose a number into base B little-endian limbs.

Gate structure is like \figref{fig:range-check-u32}.

\begin{figure}[!ht]
    \centering
    \includegraphics[width=0.5\textwidth]{gates/range_check_u32.jpeg}
    \caption{U32RangeCheckGate}
    \label{fig:range-check-u32}
\end{figure}

Constraints for each input\_limb:
\begin{itemize}
    \item Each input\_limb consists of its aux\_limbs. -- 1 constraint with degree 1.
    \begin{lstlisting}[language=rust]
let computed_sum = reduce_with_powers(&aux_limbs, base);
constraints.push(computed_sum - input_limb);
    \end{lstlisting}
    \item aux\_limbs range check. -- 16 (aux\_limbs) constraints with degree BASE ($(x-0)(x-1)\cdots(x-\text{BASE}+1)$).
\end{itemize}

In summary, there are 17 constraints per input limbs, a total of $\text{num\_input\_limbs} \times 17$ constraints. 
Degree of the gate equals BASE for range check.

\paragraph{subtraction\_u32}

\hspace*{\fill}

\indent U32SubtractionGate is a gate to perform subtraction on 32-bit limbs: given `x', `y', and `borrow', it returns 
the result $x - y - \text{borrow}$ and, if this underflows, a new `borrow'.

The gate structure is like \figref{fig:subtraction-u32}.

\begin{figure}[!ht]
    \centering
    \includegraphics[width=0.6\textwidth]{gates/subtraction_u32.jpeg}
    \caption{U32SubtractionGate}
    \label{fig:subtraction-u32}
\end{figure}

Constraints for each operation:
\begin{itemize}
    \item Constrain the calculation. -- 1 constraint with degree 2.
    \begin{lstlisting}[language=rust]
let result_initial = input_x - input_y - input_borrow;
...
constraints.push(output_result - (result_initial + base * output_borrow));
    \end{lstlisting}
    \item Limbs range check. -- 16 (limbs) constraints with degree 4. (limbs are all 2-bits)
    \item Constrain limbs for res. -- 1 constraint with degree 1.
    \item Constrain borrow\_out to be one bit. -- 1 constraint with degree 1.
\end{itemize}

In summary, there are 19 constraints for each operation. The degree of the gate is 4 which is needed by the 4-bits limbs range check.

\subsubsection{Gadgets}



\paragraph{biguint-add}

\begin{enumerate}
    \item \verb|Target|: Implement the addition of two biguints.
    \item \verb|Constraints logic|: 
        \begin{itemize}
            \item Equation for gates;
            \item Sumcheck between output and limbs;
            \item Rangecheck for limbs.
        \end{itemize}
    \item \verb|Circuit layout|: See \figref{fig:biguint-add-circuit-layout}.
    \item \verb|Trace layout|: See \figref{fig:biguint-add-trace-layout}.
    \item \verb|Constraints info and costs|
    \begin{itemize}
        \item gate type num: 1 (U32AddManyGate)
        \item gate ops num: limbs-num
        \item gate instance num: ceil(limbs-num / gate.ops)
        \item copy-constraints: limbs-num * 4
        \item max-degree: 4 (\verb|1 << limb-bits|)
    \end{itemize}
\end{enumerate}

\begin{figure}[!ht]
    \centering
    \includegraphics[width=0.6\textwidth]{biguint-add-circuit-layout.jpg}
    \caption{biguint-add circuit layout}
    \label{fig:biguint-add-circuit-layout}
\end{figure}
 
\begin{figure}[!ht]
    \centering
    \includegraphics[width=0.6\textwidth]{biguint-add-trace-layout.jpg}
    \caption{biguint-add trace layout}
    \label{fig:biguint-add-trace-layout}
\end{figure}
\paragraph{biguint-sub}

\subparagraph{Target}
Implement the substraction of two biguints.

\subparagraph{Constraints logic}
\begin{itemize}
    \item Equation for gate;
    \item Sumcheck for ouptput;
    \item Rangecheck for limbs.
\end{itemize}

\subparagraph{Circuit layout}
See \figref{fig:biguint-sub-circuit-layout}.
\begin{figure}[!ht]
    \centering
    \includegraphics[width=0.5\textwidth]{biguint-sub-circuit-layout.jpg}
    \caption{biguint-sub circuit layout}
    \label{fig:biguint-sub-circuit-layout}
\end{figure}

\subparagraph{Trace layout}
See \figref{fig:biguint-sub-trace-layout}.
\begin{figure}[!ht]
    \centering
    \includegraphics[width=0.5\textwidth]{biguint-sub-trace-layout.jpg}
    \caption{biguint-sub trace layout}
    \label{fig:biguint-sub-trace-layout}
\end{figure}

\subparagraph{Constraints info and costs}
\begin{itemize}
    \item constraints-num: $6 \times (3 + 32 / 2) = 114$
    \item copy-constraints: $16$
    \item max-degree: $4$
    \item wires-num: $6 \times (5 + 16) = 126$
\end{itemize}

\subparagraph{Questions}
\begin{itemize}
    \item Why not make rangecheck constraint for inputs?
    \item Could try to use the same constraint with add-gate.
\end{itemize}

\paragraph{biguint-mul}

\subparagraph{Target}
Implement the multiplication of two biguints.

\subparagraph{Constraints logic}
\begin{itemize}
    \item Compute mul-factors first, use U32ArithmeticGate;
    \item Add mul-factors from low bits, use U32AddManyGate.
\end{itemize}

\subparagraph{Process layout}
See \figref{fig:biguint-mul-layout}
\begin{figure}[!ht]
    \centering
    \includegraphics[width=0.6\textwidth]{biguint-mul-layout.jpg}
    \caption{biguint-mul layout}
    \label{fig:biguint-mul-layout}
\end{figure}

\subparagraph{Constraints info and costs}
\begin{itemize}
    \item Gate type num: 4 (U32ArithmeticGate, U32AddManyGate(num-addends: 4), U32AddManyGate(num-addends: 6), U32AddManyGate(num-addends: 8))
    \item Gate instance num: 9
    \item U32ArithmeticGate num: 6
    \item U32AddManyGate num: 3
    \item copy-constraints: $18 \times 3 + (4 + 6 + 8) \times 2 + 9 = 99$
    \item max-degree: 4
\end{itemize}

\paragraph{biguint-div}

Note that div-rem has the same constraints logic with div

\subparagraph{Target}
Implement the division of two biguints.

\subparagraph{Constraints logic}
\begin{itemize}
    \item Not implement div-algrithem directly;
    \item Use nondeterministic feature to check div-logic;
    \item Check \verb|div * b + rem = a|;
    \item Check \verb|rem < b|.
\end{itemize}

\subparagraph{Process layout}
See \figref{fig:biguint-div-layout}.
\begin{figure}[!ht]
    \centering
    \includegraphics[width=0.6\textwidth]{biguint-div-layout.jpg}
    \caption{biguint-div layout}
    \label{fig:biguint-div-layout}
\end{figure}

\subparagraph{Constraints info and costs}
\begin{itemize}
    \item Gate type num: 5 (U32ArithmeticGate, U32AddManyGate (num-addends: 3), U32AddManyGate (num-addends: 4), ComparisionGate, ArithmeticGate)
    \item Gate instance num: $3 + 3 + 4 + 3 = 13$
    \item U32ArithmeticGate num: 3
    \item U32AddManyGate num: 3
    \item ComparisionGate num: 4
    \item ArithmeticGate num: 3
    \item copy-constraints: $3 \times 8 + 4 + 5 + 4 + 4 \times 6 + 7 + 1 + 26 + 5 = 100$
    \item max-degree: 4
\end{itemize}

\subsubsubsection{biguint-cmp}

\begin{enumerate}
    \item \verb|Target|: Implement the comparison of two biguints.
    \item \verb|Constraints logic|
    \begin{itemize}
        \item Split the input to many limbs, such that: \verb|limbs_num = bits / chunks|;
        \item Execute comparison for low bits limbs;
        \item Ensure that the result is determined by the highest limbs which are not equal.
    \end{itemize}
    \item \verb|Process layout|: See \figref{fig:biguint-cmp-layout}.
    \item \verb|Circuit layout|: See \figref{fig:biguint-cmp-circuit-layout}.
    \item \verb|Constraints info and costs|:
    \begin{itemize}
        \item Gate type num: 2 (ComparisionGate, ArithmeticGate)
        \item Gate instance num: $4 \times 2 + 3 = 11$
        \item ComparisionGate num: 8
        \item ArithmeticGate num: 3
        \item copy-constraints: $(4 + 9) \times 4 + 1 = 53$
        \item max-degree: 4
    \end{itemize}
\end{enumerate}

\begin{figure}[!ht]
    \centering
    \includegraphics[width=0.6\textwidth]{biguint-cmp-layout.jpg}
    \caption{biguint-cmp layout}
    \label{fig:biguint-cmp-layout}
\end{figure}

\begin{figure}[!ht]
    \centering
    \includegraphics[width=0.6\textwidth]{biguint-cmp-circuit-layout.jpg}
    \caption{biguint-cmp circuit layout}
    \label{fig:biguint-cmp-circuit-layout}
\end{figure}


\subsubsubsection{non-native-add}

\begin{enumerate}
    \item \verb|Target|: Check the additional relation among three non-native target objects.
    \item \verb|Constraints logic|:
    \begin{itemize}
        \item Check equation for gadget: \verb|a + b = c + modular * overflow|;
        \item Check that ``c < modular''.
    \end{itemize}
    \item \verb|Process layout|: See \ref{fig:non-native-add-layout}.
    \item \verb|Constraints info and costs|:
    \begin{itemize}
        \item gadget biguint-add num: 2
        \item gadget biguint-mul-by-bool num: 1
        \item gadget biguint-cmp num: 1
        \item gate type num: 3 (U32AddManyGate, ComparisonGate, ArithmeticGate)
        \item gate instance num: 23 = 3 (U32AddManyGate) + 16 (ComparisonGate) + 2 (ArithmeticGate (1,0)) + 1 (ArithmeticGate(1,-1)) + 1 (ArithmeticGate(1,1))
        \item copy-constraints: 186 = 32 * 2{biguint-add} + 9{biguint-mul-by-bool} + 9 + (4 + 9) * 8{biguint-cmp} = 186
    \end{itemize}
\end{enumerate}

\begin{figure}[!ht]
    \centering
    \includegraphics[width=0.6\textwidth]{nonnative-add-layout.jpg}
    \caption{non-native-add layout}
    \label{fig:non-native-add-layout}
\end{figure}

\subsubsubsection{non-native-sub}

\begin{enumerate}
    \item \verb|Target|: Check the substrate relation among three non-native target objects.
    \item \verb|Constraints logic|:
    \begin{itemize}
        \item Check equation for gadget: \verb|diff + b = a + modular * overflow|;
        \item Check that ``overflow is bool'';
        \item Check that ``diff.limbs is range U32''.
    \end{itemize}
    \item \verb|Process layout|: See \figref{fig:non-native-sub-layout}
    \item \verb|Constraints info and costs|:
    \begin{itemize}
        \item gadget biguint-add num: 1
        \item gadget biguint-sub num: 1
        \item gadget biguint-mul-by-bool num: 1
        \item gadget u32rangecheck num: 1
        \item gadget assert-bool num: 1
        \item gate type num: 4 (U32AddManyGate, U32SubtractionGate, U32RangeCheckGate, ArithmeticGate)
        \item gate instance num: 7 = 2 (U32AddManyGate) + 2 (U32SubtractionGate) + 1 (U32RangeCheckGate) + 1 (ArithmeticGate(1,0)) + 1 (ArithmeticGate(1,-1))
        \item copy-constraints: 89 = 32 (biguint-add num) + 27 (U32SubtractionGate) + 9(biguint-mul-by-bool) + 8 (u32rangecheck) + 4 (assert-bool) + 9
    \end{itemize}
\end{enumerate}

\begin{figure}[!ht]
    \centering
    \includegraphics[width=0.6\textwidth]{nonnative-sub-layout.jpg}
    \caption{non-native-sub layout}
    \label{fig:non-native-sub-layout}
\end{figure}

\subsubsubsection{non-native-mul}

\begin{enumerate}
    \item \verb|Target|: Check the multiplication relation among three non-native target objects.
    \item \verb|Constraints logic|:
    \begin{itemize}
        \item Check equation for gadget: \verb|a * b = prod + modular * overflow|;
        \item Check that ``overflow.limb is U32'';
        \item Check that ``prod.limb is U32''.
    \end{itemize}
    \item \verb|Process layout|: See \figref{fig:non-native-mul-layout}.
    \item \verb|Constraints info and costs|:
    \begin{itemize}
        \item gadget biguint-add num: 1
        \item gadget biguint-mul num: 2
        \item gadget u32rangecheck num: 2
        \item gate type num: 9 = 7 (U32AddManyGate\{3,5,7,9,11,13,15\}) + 1 (U32RangeCheckGate) + 1 (U32ArithmeticGate)
        \item gate instance num: 37 = 2 (u32rangecheck) + 8 (biguint-mul: constant-input) + 22 (biguint-mul) + 1 + 3 (biguint-add)
        \item copy-constraints: 583 = 8 * 2 (u32rangecheck) + 3 * 3 + (4 + 6 + 8 + 10 + 12 + 14 + 16) * 4 + (8 * 8) * 3 + 17 * 4 + 18
    \end{itemize}
\end{enumerate}

\begin{figure}[!ht]
    \centering
    \includegraphics[width=0.6\textwidth]{nonnative-mul-layout.jpg}
    \caption{non-native-mul layout}
    \label{fig:non-native-mul-layout}
\end{figure}

\paragraph{nonnative-inv}

\subparagraph{Target}
Check the modular inverse relation among three nonnative target objects.

\subparagraph{Constraints logic}
\begin{itemize}
    \item Check equation for gadget: \verb|a * inv_a = 1 + modular * div|.
\end{itemize}

\subparagraph{Process layout}
See \figref{fig:nonnative-inv-layout}.
\begin{figure}[!ht]
    \centering
    \includegraphics[width=0.8\textwidth]{nonnative-inv-layout.jpg}
    \caption{nonnative-inv layout}
    \label{fig:nonnative-inv-layout}
\end{figure}

\subparagraph{Constraints info and costs}
\begin{itemize}
    \item gadget biguint-add num: 1
    \item gadget biguint-mul num: 2
    \item gate type num: 8 = 7(U32AddManyGate\{3,5,7,9,11,13,15\}) + 1(U32ArithmeticGate)
    \item gate instance num: 56 = (8 * 8 + 2) * 2 / 3 + 5(U32AddManyGate{3}) + 5 + 2(U32AddManyGate{15})
    \item copy-constraints: 762 = (8 * 8 + 2) * 2 * 3 + 21 * 4 + (6 + 8 + 10 + 12 + 14) * 4 + 4 * 16 + 18
\end{itemize}


\paragraph{curve-add}

\subparagraph{Target}
Implement the addition of two different curve points. this is a incomplete addition, you can refer to The halo2 Book \cite{website:halo2-book} to learn more about it.

\subparagraph{Constraints logic}
$(x_1,y_1) \ne (x_2,y_2)$, See \figref{fig:curve-add}.
\begin{figure}[!ht]
    \centering
    \includegraphics[width=0.8\textwidth]{curve-add.jpg}
    \caption{curve-add}
    \label{fig:curve-add}
\end{figure}

\subparagraph{Process layout}
See \figref{fig:curve-add-layout}
\begin{figure}[!ht]
    \centering
    \includegraphics[width=0.8\textwidth]{curve-add-layout.jpg}
    \caption{curve-add layout}
    \label{fig:curve-add-layout}
\end{figure}

\subparagraph{Constraints info and costs}
\begin{itemize}
    \item gadget-sub-nonnative num: 5
    \item gadget-add-nonnative num: 1
    \item gadget-mul-nonnative num: 3
    \item gadget-inv-nonnative num: 1
    \item gate type num: 14
        \begin{itemize}
            \item 8: U32AddManyGate\{2,3,5,7,9,11,13,15\}
            \item 1: ComparisonGate
            \item 1: ArithmeticGate
            \item 1: U32ArithmeticGate
            \item 1: U32SubtractionGate
            \item 2: U32RangeCheckGate\{0,8\}
        \end{itemize}
\end{itemize}

\paragraph{curve-double}

\subparagraph{target}
Implement the addition of two same curve points. this is an incomplete addition, you can refer to The halo2 Book \cite{website:halo2-book} to learn more about it.

\subparagraph{Constraints logic}
$(x_1,y_1) = (x_2,y_2)$. See \figref{fig:curve-add}.

\subparagraph{Process layout}
See \figref{fig:curve-double-layout}.
\begin{figure}[!ht]
    \centering
    \includegraphics[width=0.6\textwidth]{curve-double-layout.jpg}
    \caption{curve-double layout}
    \label{fig:curve-double-layout}
\end{figure}

\subparagraph{Constraints info and costs}
\begin{itemize}
    \item gadget-sub-nonnative num: 3
    \item gadget-add-nonnative num: 5
    \item gadget-mul-nonnative num: 4
    \item gadget-inv-nonnative num: 1
    \item gate type num: 14
        \begin{itemize}
            \item 8: U32AddManyGate\{2,3,5,7,9,11,13,15\}
            \item 1: ComparisonGate
            \item 1: ArithmeticGate
            \item 1: U32ArithmeticGate
            \item 1: U32SubtractionGate
            \item 2: U32RangeCheckGate\{0,8\}
        \end{itemize}
\end{itemize}

\subsubsubsection{curve-assert-valid}

\begin{enumerate}
    \item \verb|Target|: Check the point is on the curve $y^2 = x^3 + ax + b$.
    \item \verb|Process layout|: See \ref{fig:curve-assert-valid-layout}.
    \item \verb|Constraints info and costs|:
    \begin{itemize}
        \item gadget-add-nonnative num: 3
        \item gadget-mul-nonnative num: 3
        \item gate type num: 13
            \begin{itemize}
                \item 8: U32AddManyGate\{2,3,5,7,9,11,13,15\}
                \item 1: ComparisonGate
                \item 1: ArithmeticGate
                \item 1: U32ArithmeticGate
                \item 2: U32RangeCheckGate\{0,8\}
            \end{itemize}
    \end{itemize}
\end{enumerate}

\begin{figure}[!ht]
    \centering
    \includegraphics[width=0.6\textwidth]{curve-assert-valid-layout.jpg}
    \caption{curve-assert valid layout}
    \label{fig:curve-assert-valid-layout}
\end{figure}  
\subsubsubsection{curve-msm}

\begin{enumerate}
    \item \verb|Target|: Implement the multiplication scalar multiplication (MSM).
    \item \verb|Constraints logic|: See \figref{fig:curve-msm}.
    \item \verb|Constraints info and costs|:
    \begin{itemize}
        \item gate type num: 14
        \item gate instance num: 147553
    \end{itemize}
\end{enumerate}

\begin{figure}[!ht]
    \centering
    \includegraphics[width=0.6\textwidth]{curve-msm.jpg}
    \caption{curve-msm}
    \label{fig:curve-msm}
\end{figure}

\paragraph{curve-scalar}

\subparagraph{Target}
Implement the multiplication between scalar and point.

\subparagraph{Constraints logic}
See \figref{fig:curve-scalar}.
\begin{figure}[!ht]
    \centering
    \includegraphics[width=0.8\textwidth]{curve-scalar.jpg}
    \caption{curve-scalar}
    \label{fig:curve-scalar}
\end{figure}

\subparagraph{Constraints info and costs}
\begin{itemize}
    \item gate type num: 13 
    \item gate instance num: 180364          
\end{itemize}


\paragraph{ecdsa}

\begin{enumerate}
    \item \verb|Target|: Implement ecdsa verify process.
    \item \verb|Constraints logic|: See \figref{fig:ecdsa}.
    \item \verb|Constraints info and costs|:
    \begin{itemize}
        \item gate type num: 16
        \item gate instance num: 98039
    \end{itemize}
\end{enumerate}

\begin{figure}[!ht]
    \centering
    \includegraphics[width=0.6\textwidth]{ecdsa.jpg}
    \caption{ecdsa}
    \label{fig:ecdsa}
\end{figure}




\subsection{Inner Table Lookup}\label{section: inner-table-lookup}

Inner-table lookup can be thought of as a subset argument technology between columns I and T, proving that cells in column I all appear in column T at least once, and it does not matter whether the cells in column T are in I. To simplify the calculation, the columns in the STARK table are all $2^k$ rows, and when the cells of columns I and T are not $2^k$, they need to be extended.

\begin{itemize}
    \item Cells in column I can be extended with elements in column I or column T
    \item Cells in column T can only be extended with cells in column T, usually the last one
\end{itemize}

The inner-table lookup technology is to first generate the corresponding permutation columns I' and T' from columns I and T, respectively. The permutation relationship between I and I', T and T' can be checked through the permutation argument during the proving phase:
$$Z(\omega X) \cdot (I'(X) + \beta) \cdot (T'(X) + \gamma) - Z(X) \cdot (I(X) + \beta) \cdot (T(X) + \gamma) = 0$$
$$1 - Z(\omega^0) = 1 - Z(\omega^{k}) = 0$$

\noindent Then sort columns I' and T' as follows:

\begin{enumerate}
    \item Column I' is ordered by a sorting algorithm so that cells of the same value are adjacent to each other.
    \item Column T' is ordered so that the first row in a sequence of the same values in I' has the same value in T' at the same position.
\end{enumerate}

\noindent Now, we'll enforce either $I'_{i} = T'_{i}$ or that $I'_{i}=I'_{i-1}$, using the rule:
$$(I'(X) - T'(X)) \cdot (I'(\omega X) - I'(X)) = 0$$
$$I'(\omega^0) - T'(\omega^0) = 0$$

These constraints effectively force every element in I' (and thus I) to equal at least one element in T' (and thus T).
\subsection{Cross Table Lookup}\label{section: cross-table-lookup}

The goal of cross table lookup is different from that of inner-table lookup, which is not only a subset argument, but a mix of subset argument and permutation argument.

Cross table lookup indicates the multi-table lookup in the \href{https://eprint.iacr.org/2020/315.pdf}{plookup paper}, that is, the data in a looked table comes from multiple looking tables. When implementing STARK tables, Starky should define all the STARK cross table lookup relationships, that is, which other STARK tables each STARK table needs to look from, and which columns and rows are involved(filtered) in the looked table and looking table respectively during the lookup. Each STARK table needs to perform cross table lookups, so there are multiple CrossTableLookups in Starky, and each cross table lookup instance consists of a looked table and multiple looking tables.

We generate a Z(X) polynomial from calculating the permutation product as the grand product argument polynomial for the permutation argument in a cross table lookup instance, and then enforce that they are permutations using a permutation argument with Z(X) polynomial

\begin{enumerate}
    \item Initialize cross table lookup instances and random numbers $\beta, \gamma$
    \item Calculate the partial products of the looking table and looked table, compressing the values of multiple columns into one:$$C = \Pi(c_i * \beta^i) + \gamma$$
    \item Calculate Z(X) polynomial from partial products $$Z(\omega X) = Z(X)\cdot (C \cdot filter + 1 - filter)$$
    \item Verify that the product of the last element of the Z(X) polynomials of all looking tables is equal to the last element of the Z(X) of the looked table
    $$\Pi_{i=0} Z^{looking}_{i}(\omega^{k_i-1}) = Z^{looked}(\omega^{k_{looked}-1})$$
\end{enumerate}

Cross table lookup enforces STARK tables are correctly generated from the program, and reduces the rows in CPU STARK table.

\subsection{GPU Acceleration}\label{section: gpu-acceleration}
\subsection{FPGA Acceleration}\label{section: fpga-acceleration}

The NTT acceleration design consists of two main parts: the NTT engine and the DMA data transfer engine. The NTT engine performs calculations on given-length data points, while the DMA data transfer engine efficiently exchanges data between the NTT engine and DDR.

\begin{figure}[ht]
  \centering
  \includegraphics[width=0.8\textwidth]{The design of NTT implemented in FPGA.jpg}
  \caption{The design of NTT implemented in FPGA}
  \label{fig:The design of NTT implemented in FPGA}
\end{figure}

\subsubsection{NTT engine}

In this design, the NTT engine consists of multiple parallel-running NTT modules. The maximum supported data length for a single NTT module is $2^{12}$. A large number of modular multiplication units is connected to the output of each NTT module to support the extension of two-dimensional and higher-dimensional NTTs. Additionally, to improve scalability performance, variable-length NTT calculations can be achieved through parameter configuration.

In practical applications, processing in parallel with multiple NTT modules can greatly improve computational performance. For a single NTT module, one data read and one storage operation are required per clock cycle. When 8 NTTs are processed in parallel as a unit, assuming a working clock of 100MHz, the total data access bandwidth required is 8 * 8 byte/point * (1x read + 1x write) * 100MHz = 11.92 GB/sec. That is, the calculation rate of NTT is 8 * 100MHz / $2^{24}$ / 2= 25 times/sec. For the AWS FPGA platform, the user logic can use three DDR4 channels with a frequency of 2100MHz. The total supported simultaneous access bandwidth is 3 * 8 byte/DDR * 2100MHz = 46.9 GB/sec. Therefore, the maximum supported NTT calculation rate is 25 * 46.9 / 11.92 = 98 times/sec.


\paragraph{NTT calculation module}

To support an NTT length of $2^{24}$, it is better to consider the $2^{24}$ points as a 2D array with a size of $2^{12}$ points by $2^{12}$ points. Using the Gentleman-Sande algorithm, the first pass through the engine computes $2^{12}$ NTTs column-wise over the 2D array using the carefully designed NTT module with a length of $2^{12}$. A recursive NTT Algorithm, is adopted, which contains 12 stages of butterfly with varying kernel sizes. Therefore, users can adjust the sizes of both columns and rows in the 2D array by setting parameters to meet FPGA running requirements. Finally, the output of the NTT block performs modular multiplication according to the 2D NTT algorithm.

\begin{figure}[ht]
  \centering
  \includegraphics[width=1\textwidth]{Various-size NTT block.jpg}
  \caption{Various-size NTT block (MM: modular multiplication)}
  \label{fig:NTT_module}
\end{figure}

\paragraph{NTT kernel}
The NTT block contains several kernels, each of which is mainly composed of a butterfly unit and two RAMs. The butterfly unit is carefully designed to take maximum advantage of the FPGA DSPs, especially by utilizing the tricks of the Goldilocks field multiply to improve computational efficiency over modular reduction. One of the RAMs is used to buffer the input data, while the other stores the twiddle factors required for butterfly calculation.


\begin{figure}[ht]
  \centering
  \includegraphics[width=0.5\textwidth]{NTT kernel.jpg}
  \caption{NTT kernel (TF: twiddle factor; BF: Butterfly)}
  \label{fig:NTT_kernel}
\end{figure}

\paragraph{Generation of twiddle factors}

To realize the 2D NTT, the outputs of the 1st pass of the NTT module should be multiplied by the corresponding twiddle factors, as shown in Fig.2. Since the twiddle factors required for the NTT output of different columns of the 2D array are different, these twiddle factors must be updated for every column. To reduce the necessary DDR bandwidth, a combination of LUT and computation is used to generate these twiddle factors. A LUT provides the $2^{12}$ twiddles needed for modular multiplication before the 2nd pass. The updating twiddle factor is computed by multiplying the LUT twiddles with an accumulated running twiddle factor and then writing back to the LUTs. Additionally, the twiddle factors needed in the NTT block are provided by LUT and loaded once before performing the NTT. Therefore, the total multiplication/reduction unit count per chain is 13: 11 for the 12 butterfly stages and 2 for each twiddle updating and modulation multiplication out of the NTT block.

\paragraph{Bit Reversal}
The NTT engine performs bit reversal during NTT computation, which is achieved by bit reversing each pass of the NTT at the end of the block (as shown in the above NTT blocks). A final step is required to complete the bit reversal across the two passes. This can be achieved by writing back the results of the 2nd pass using the column-major access pattern instead of the row-major access pattern.


\subsubsection{DMA Engine}

The sole responsibility of the DMA Engine is to exchange points between the NTT Engine and the memory outside of the FPGA at the NTT Engine's burst processing rate. With the NTT Engine capable of consuming and producing 8 points per cycle when 8 NTT blocks are working concurrently, and an NTT Engine clock rate nearing 300MHz, the feed and sink rates are both 17.9 GB/sec. The FPGA's DDR4 memory, with a peak bandwidth of 46.9 GB/sec, can support the combined feed and sink bandwidth need of 35.8 GB/sec while also providing enough storage for $2^{24}$ points.
  In addition, to fully utilize the bandwidth of DDR4 and continuously provide high data bandwidth to the NTT engine, a large data buffer RAM is introduced in the DMA engine to maintain a high data throughput rate for the AXI4 interface.

\subsubsection{Simulation}

To verify the feasibility of the designed NTT acceleration scheme, it was implemented in Vivado using Verilog HDL language and both functional and timing simulations were performed. The simulation results are shown in the following figure.

\begin{figure}[h]
  \centering
  \includegraphics[width=1\textwidth]{NTT simulation.jpg}
  \caption{NTT simulation}
  \label{fig:NTT_Simu}
\end{figure}

In the Figure: the signals WeightEn and WeightIn are configurations for the twiddle factors of the NTT module; WeightEnStep2 and WeightInStep2 are twiddle factors for the output of large number modular multiplication in the NTT module; signals WeightIntervalEn and WeightIntervalIn are used to accumulate twiddle factors when generating them; DataIn and DataEn are the input data for the first dimension of the NTT, while DataOut\_1st and DataValid\_1st are the computed results outputted; DataIn\_2nd and DataEn\_2nd are the input data for the second dimension of the NTT, and DataOut\_2nd and DataValid\_2nd are the computed results outputted; DataCalCmp is used for comparing the computed NTT results with the theoretical ones.

In the implementation, 8-way parallel NTT acceleration is used. According to the timing, the NTT engine is respectively configured with LUT twiddle factors, output large number modular multiplication LUT twiddle factors, and cumulative twiddle factors. To compute a 4096-point NTT, the size of the 2D array is 16 * 4. Based on the comparison signal DataCalCmp, the results obtained from the 2D NTT operation are consistent with the theoretical values. When performing the second 4096-point NTT acceleration, it is only necessary to reload the output large number modular multiplication LUT twiddle factors.
