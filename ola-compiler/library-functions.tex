\subsection{Library Functions}

\subsubsection{Ola-lang Library Goal}

\begin{itemize}
    \item efficiency

    The goal of the Ola-lang high-level language library is to provide a set of high-level APIs that can be used to quickly develop applications.
The library provides commonly used functions and modules, such as Ola Standard Library, integer type operations, math calculations,
which can greatly improve the development efficiency of programmers.

    In addition, the library also will provide third-party modules that can be used to extend the functionality of the library.
    \item  integer types functions

    Another goal is that since the native type of the ola language is the field type,
operators of other types such as u32 and other integer types need to be simulated and converted through field-type operations to Implement their functions.
\end{itemize}

\subsubsection{Language Features}
    For functionality, efficiency, and performance, our library functions usually combine Program, Builtin, and Prophet these three language features in their implementation.

    The ola language uses standard syntax such as operations and control flow to describe the algorithmic process of library functions.

    Ir generates the standard llvm ir, and also generates some builtin and prophet functions; these two types of functions generate calls and declarations but do not generate function bodies for function definitions.

    The assembly code generates the standard olavm instructions; the builtin functions are converted to standard instructions such as rangecheck assert;
the prophet functions to generate a label in the body of the program, while the function body, input, and output information are expanded to facilitate interpretation and execution in the VM.
\begin{itemize}
    \item Program

    The ola language describes the algorithmic process, Ir generates llvm ir containing builtin and prophet libFunctions, and The assembly code generates the standard olavm instructions and prophets.
    \item Builtin

    The ola language is non-sensitive, Ir generates builtin libFunctions such as rangecheck and assert.
    \item Prophet

    The ola language is non-sensitive, Ir generates prophet libFunctions such as sqrt, div, and mod, and verifies the result.
The assembly code generates prophet APIs for the VM such as function body, inputs, and outputs. and load psp to get the result.
\end{itemize}

\subsubsection*{Example of Assembly Pseudocode for a Typical Library Functions}

Its pseudo-code format is as follows:
\begin{lstlisting}[language={}]
{
  "program": " // be directly executed by vm
    funcName:
    .LBL{funcNum}_{bbNum}:
      insts // such as mov add mload mstore call general instructions
      .PROPHET{funcNum}_{prophetNum}:  // bind to prophet label
      mov r0 psp  // interact with prophet read-only memory, get return value from prophet pointer
      mload r0 [r0,0]  // used returned r0 as indexed addressing
      rangecheck  // rangecheck builtin inst
      insts
      assert  // assert builtin inst
      insts
      return",
  "prophets": [  // be indirectly executed by the embedded interpreter in vm
    {
      "label": ".PROPHET{funcNum}_{prophetNum}", // according with prophet label in program
      "code": "%{\n    entry() {\n        cid.y = prophet_function_name(cid.x);\n    }\n%}",  // prophet body, like solidity syntax
      "inputs": [  // prophet params
        "cid.x"
      ],
      "outputs": [  // prophet return values
        "cid.y"
      ]
    }
  ]
}
\end{lstlisting}
The comments in json asm describe the program, builtin, and prophet and their interaction in compiler output olavm assembly code.

\begin{itemize}
    \item Program

The program will be directly executed by VM after assembler encoding. It contains mainly instructions, rangecheck and assert builtin instructions,
and call prophet displayed as prophet label format. It interacts with prophet read-only memory and gets a return value from the prophet pointer.
Then use the return value register as indexed addressing to go on next program logic.
    \item Prophets

The other part as the prophets list will be indirectly executed by an embedded interpreter in VM.
The API field contains labels according to the prophet label in the program, prophet body which code is like solidity syntax,
input as prophet params and outputs as prophet return values indirectly passed to the program.
\end{itemize}

\subsubsection{U32 Library Functions List}
    Take u32 integer type libFunctions for example, its implementation in the language and the front and back of the compiler is rough as follows table \ref{table:u32-libFunctions-list}:

    \begin{table}[!htp]
        \resizebox{\textwidth}{!}{
            \begin{tabular}{|c|c|c|c|c|}
                \hline
                \textit{Functions}  & \textit{Impl Notes} & \textit{Ola-lang} & \textit{IR} & \textit{Asm}  \\ \hline
                u32\_add  & rangecheck & a+b &  &  \\ \hline
                u32\_sub  &  rangecheck & a-b &  &  \\ \hline
                u32\_div  & \makecell{rangecheck \\ div prophet \\ assert} &  a/b & \makecell{q=prophet\_div(a, b); \\ assert} & \makecell{passing params a,b;\\prophet\_div body;\\get q from psp register;\\assert} \\ \hline
                u32\_mod  & \makecell{rangecheck \\ mod prophet \\ assert} &  a\%b & \makecell{r=prophet\_mod(a, b); \\ assert} & \makecell{passing params a,b;\\prophet\_div body;\\get r from psp register;\\assert} \\ \hline
                u32\_increment  &  & a++ &  &  \\ \hline
                u32\_decrement  &  & a-- &  &  \\ \hline
                u32\_or  & conditional branch & a||b & sep to multi brcond opcode & pattern icmp+brcond \\ \hline
                u32\_and  & conditional branch & a\&\&b & sep to multi brcond opcode & pattern icmp+brcond \\ \hline
                u32\_bitwise\_or  & custom to isa inst & a|b & bitwise or ir opcode & bitwise builtin or inst \\ \hline
                u32\_bitwise\_and  & custom to isa inst & a\&b & bitwise or ir opcode & bitwise builtin and inst \\ \hline
                u32\_equal  &  & a==b &  &  \\ \hline
                u32\_not\_equal  &  & a!=b &  &  \\ \hline
                u32\_more  &  & a>b &  &  \\ \hline
                u32\_more\_equal &  & a>=b &  &  \\ \hline
                u32\_less  &  & a<b &  &  \\ \hline
                u32\_less\_equal  &  & a<=b &  &  \\ \hline
                u32\_shift\_left  & \makecell{i=2\^b; \\ a=a*i; \\ u32\_power} & a<<b &  &  \\ \hline
                u32\_shift\_right  & \makecell{i=2\^b; \\ a=a/i; \\ u32\_power \\ div prophet} & a>>b &  &  \\ \hline
                u32\_not  & a==0 & \!a &  &  \\ \hline
                u32\_complement  & u32\_max-a & -a &  &  \\ \hline
                u32\_power  & forloop & a\^b &  &  \\ \hline
            \end{tabular}}
        \caption{U32 LibFunctions List}
        \label{table:u32-libFunctions-list}
    \end{table}