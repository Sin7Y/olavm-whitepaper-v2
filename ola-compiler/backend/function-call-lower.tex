\subsubsection{ABI Lower: Lowering Function Call}
    
Ola Procedure Call Standard (OPCS) are as follows:

The stack initialization points to the first address of the frame stack after the \texttt{fp} register is loaded.
    
The address will be increased when the \texttt{call} instruction is executed later.
When the \texttt{ret} instruction is executed, the \texttt{fp} register points to the address and falls back.
    
    
The Calling process is as follows:
\begin{itemize}
    \item call label

Caller use \texttt{call} instruction to call a callee as \texttt{call functionLabel}, and \texttt{fp} points to the new frame.\par
The \texttt{pc} address returned by the callee is placed in \texttt{fp-1} which is detected by VM but not visible by the compiler backend.\par
Its instructions pattern are as follows:
\begin{lstlisting}[language={}]
main:
.LBL0_0:
  ...
  call foo
  ...
foo:
.LBL1_0:
  ...
\end{lstlisting}
    \item function address

The address pointed to by \texttt{fp} before the function call is placed in \texttt{fp-2} as \texttt{mstore [r8,-2] r8}.\par
Its instructions pattern are as follows:
\begin{lstlisting}[language={}]
mstore [r8,-2] r8
\end{lstlisting}
    \item passing arguments

Function parameter processing: the first three input parameters are placed in the three registers \texttt{r1}, \texttt{r2}, and \texttt{r3}.
If there are more than 3 parameters, start with the fourth input parameter and descend accordingly in \texttt{fp-3}, \texttt{fp-4}, \texttt{...}. \par
Its instructions pattern are as follows:
\begin{lstlisting}[language={}]
mov r1 vreg1
mload r2 [r8,offset]
mov r3 vreg2
\end{lstlisting}
    \item  local variables

Local variables inside the function start at old \texttt{fp}, and their addresses are stored incrementally.

The single return value is stored in \texttt{r0}. If there are multi return values, it needs to be returned by a memory pointer that return the package data.\par
Instruction pattern for single return value is as follows:
\begin{lstlisting}[language={}]
mov r0 vreg3
\end{lstlisting}
\end{itemize}

The call stack frames layout is as follows \ref{fig:ola-lang-backend-functioncall}:

\begin{figure}[!htp]
    \centering
    \includegraphics[width=0.8\textwidth]{ola-lang-backend-functioncall.jpg}
    \caption{Ola-lang Function Call Stack Frames}
    \label{fig:ola-lang-backend-functioncall}
\end{figure}

For prophet library functions, its instructions pattern as:
\begin{lstlisting}[language={}]
.PROPHET{funcNum}_{prophetNum}:  // bind to prophet label
mov r0 psp  // interact with prophet read-only memory, get return value from prophet pointer
mload r0 [r0,0]  // used returned r0 as indexed addressing
\end{lstlisting}

First \texttt{.PROPHET} label binds to the prophet instance in assembly output.
Then the program interacts with prophet read-only memory, get the return value from prophet pointer \texttt{[psp]} and write the result into \texttt{r0}.
At last, we use \texttt{r0} as indexed addressing to load return values from prophet memory.
