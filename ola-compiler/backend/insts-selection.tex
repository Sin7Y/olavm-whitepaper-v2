\subsubsection{Instruction Selection: Match Pattern from IR to Target Instruction}

It finds opcode and operands of instruction according to the given pattern, then select the best target single or multi instructions to match with it.

Several common patterns briefly are as follows table:
\begin{table}[!ht]
    \resizebox{\textwidth}{!}{
        \begin{tabular}{|c|c|}
            \hline
            \textit{Pattern Type} & \textit{Description} \\ \hline
            Alloca  & params and vars allocation, selected to memmory operations \\ \hline
            IntBinary & bianary operator \\ \hline
            Load & memory load, containing base or offset \\ \hline
            Store & memory store, containing base or offset  \\ \hline
            Call & A(caller) call B(callee) \\ \hline
            Return & B(callee) return to A(caller) \\ \hline
            Branch & branch control flow, selected to jump operations \\ \hline
            Conditional Branch & conditional branch control flow, selected to compare and jump operations \\ \hline
        \end{tabular}}
    \caption{Instruction Selection Pattern}
    \label{table:Instruction-pattern}
\end{table}

Let's take the Conditional Branch more specifically, for example, it's target instructions are as follows table \ref{table:conditional-branch-pattern}:
\begin{table}[!ht]
    \resizebox{\textwidth}{!}{
        \begin{tabular}{|c|c|c|c|}
            \hline
            \textit{Operator}  & \textit{Reg and Imm} & \textit{Reg and Reg} & \textit{Cycles}  \\ \hline
            == & \makecell{mov tmpReg imm \\ eq tmpReg regA tmpReg \\ cjmp tmpReg labelTrue} & \makecell{eq tmpReg regA regB \\ cjmp tmpReg labelTrue} & \makecell{3inst + 2reg \\ 2inst + 3reg} \\ \hline
            < & \makecell{mov tmpReg1 imm \\ gte tmpReg1 tmpReg1 regA \\ neq tmpReg2 tmpReg1 regA \\and tmpReg2 tmpReg2 tmpReg1 \\ cjmp tmpReg2 labelTrue} & \makecell{gte tmpReg1 regB regA \\ neq tmpReg2 regA regB \\ and tmpReg2 tmpReg1 tmpReg2 \\ cjmp tmpReg2 labelTrue} & \makecell{5inst + 3reg \\ 4inst + 4reg} \\ \hline
            <= & \makecell{mov tmpReg imm \\ gte tmpReg tmpReg regA \\ cjmp tmpReg labelTrue} & \makecell{gte tmpReg regA regB \\ cjmp tmpReg labelTrue} & \makecell{3inst + 2reg \\ 2inst + 3reg} \\ \hline
            > & \makecell{mov tmpReg1 imm \\ gte tmpReg1 regA tmpReg1 \\ neq tmpReg2 tmpReg1 regA \\and tmpReg2 tmpReg2 tmpReg1 \\ cjmp tmpReg2 labelTrue} & \makecell{gte tmpReg1 regA regB \\ neq tmpReg2 regA regB \\ and tmpReg2 tmpReg1 tmpReg2 \\ cjmp tmpReg2 labelTrue} & \makecell{5inst + 3reg \\ 4inst + 4reg} \\ \hline
            >= & \makecell{mov tmpReg imm \\ gte tmpReg regA tmpReg \\ cjmp tmpReg labelTrue} & \makecell{gte tmpReg regA regB \\ cjmp tmpReg labelTrue} & \makecell{3inst + 2reg \\ 2inst + 3reg} \\ \hline
            != & \makecell{mov tmpReg imm \\ neq tmpReg regA tmpReg \\ cjmp tmpReg labelTrue} & \makecell{neq tmpReg regA regB \\ cjmp tmpReg labelTrue} & \makecell{3inst + 2reg \\ 2inst + 3reg} \\ \hline
        \end{tabular}}
    \caption{Conditional Branch Selection Pattern}
    \label{table:conditional-branch-pattern}
\end{table}