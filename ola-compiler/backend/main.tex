\subsection{Ola Compiler Backend: from LLVM IR to OlaVM assembly}

Ola compiler backend bridge ir structure of module, function, and isa related callconv, registers and instructions.
Its main features are code generation related lower and optimization-related pass.

Target ISA main contains custom TargetInst, Register(RegisterClass/RegisterInfo), lower and modulepasses, callconv, datalayout information.

Module in addition to inheritance ir parsed out Module, the description of its Function and ir differ significantly.
Data information of BasicBlock in the instructions for Target Instruction, the register contains VRegs and RegUnit two categories, and contains a vregtoinsts mapping.
At the same time, Inst in Layout is referred to TargetInst. Note that the memory access operations of parameters, local variables, etc. are described by the structure of slot(ptr+offset).

The lower module provides the process of downgrading LoweringContext to Instruction, and for function calls it also requires copyargstovregs.

The pass module contains the regalloc and spiller for analyzing the liveness of the pass and the function pass.

\subsubsection{Parser: Parse LLVM IR to Module Instruction}

\begin{itemize}
    \item IR Module

IR is composed of the following multi level structures:
\begin{lstlisting}[language={}]
module -> function -> value -> types
\end{lstlisting}

(1) Module structure contains Target which contains triple and datalayout information, Function, Attribute and GlobalVariable.

(2) Function structure contains Name, Type information, Visibility, Attribute and Parameter list basic information and DataLayout.

(3) Datalayout contains the sequential logical relationship between BasicBlock and the Instruction within them.

(4) Data contains the specific Value, Instruction, BasicBlock list instances.

(5) BasicBlock is identified by BasicBlockId and consists of two parts: name and number. Each BB usually contains one predecessor and one successor
except for the entrance BB, which just only has one successor but not predecessor.

(6) Instruction is identified by BasicBlockId, InstructionId and usually consists of opcode, source operands, dest operand and type of its operation.

(7) Value contains Instruction, Argument, Constant and Inline Assembly types.

(8) Due to the characteristics of the instruction set of OlaVM, The types currently supported by the compiler are mainly \texttt{i64} and the occasional \texttt{i1}.

    \item IR Parser

The role of IR parser is to parse LLVM IR to Module Instruction.
Its parsing briefly process is as follows:

(1) Parser parse target DataLayout and Triple, the result is target data information.

(2) Parser parse attribute group, the result is attribute information of module.

(3) Parser parse local types in module, the result is registered type in module.

(4) Parser parse global variables, the result is global variables symbol table.

(5) Parser parse function which is mainly divided into arguments list and function body, the result is function structure in module instruction.

(6) Parser parse metadata, the result is metadata map in module.

Its pipeline process is as follows \ref{fig:ola-lang-backend-parser}:
\begin{figure}[!htbp]
    \centering
    \includegraphics[width=0.6\textwidth]{ola-lang-backend-parser.png}
    \caption{Ola-lang Backend Parser Pipeline}
    \label{fig:ola-lang-backend-parser}
\end{figure}
\end{itemize}

\subsubsection{Opt: opt pass on IR insts}

Pre-analysis analysis pass is mainly DominatorTree, conversion transform pass contains dce, mem2reg, sccp.

Its pipeline process is as follows:
\begin{figure}[!htbp]
    \centering
    \includegraphics[width=0.6\textwidth]{ola-lang-backend-ir-opt.jpg}
    \caption{ola-lang backend ir opt}
    \label{fig:ola-lang-backend-ir-opt}
\end{figure}
\subsubsection{ISA description: register and insts}

\begin{itemize}
    \item back-end codegen modules

bridging ir structure of module, function, and isa related callconv, register and isa, code generation related lower and optimization related pass.

TargetISA main contains custom TargetInst, Register(RegisterClass/RegisterInfo), lower and modulepasses, callconv, datalayout information.

Module in addition to inheritance Ir parsed out Module, the description of its Function and ir differ significantly.

Data information: BasicBlock in the instructions for Target Instruction, register contains VRegs and RegUnit two categories, and contains a vregtoinsts mapping.
At the same time, Inst in Layout are referred to TargetInst. Note that the memory access operations of parameters, local variables, etc. are described by the structure of slot(ptr+offset).

The lower module provides the process of downgrading LoweringContext to Instruction, and for function calls it also requires copyargstovregs.

The pass module contains the regalloc and spiller for analyzing the liveness of the pass and the function pass.

        \item back-end process

(1) Register and insts description

Register description is as below:
\begin{table}[!ht]
    \resizebox{\textwidth}{!}{
        \begin{tabular}{|c|c|c|}
            \hline
            \textit{Type}  & \textit{Description} & \textit{Register Group}  \\ \hline
            general registers & general used by program &  $[r0-r8] $ \\ \hline
            return regsiter & return value for return to caller &  $[r0] $ \\ \hline
            parameters rigsters & parameters value for passing arguments &  $ [r1, r2, r3] $ \\ \hline
            tmp registers & tmp alloc for local variables &  $[r4, r5, r6, r7]$ \\ \hline
            stack pointer & function's stack pointer &  $[r8] $ \\ \hline
            special registers & interact with vm: pc for program counter and psp for prophet pointer &   $[pc, psp] $ \\ \hline
        \end{tabular}}
    \caption{Register Description}
    \label{table:register-description}
\end{table}

Insts description is as bellow:

Opcode with register or immdiate number:
\begin{lstlisting}[language={}]
    ADDri,
    ADDrr,
    MULri,
    MULrr,

    EQri,
    EQrr,
    ASSERTri,
    ASSERTrr,

    MOVri,
    MOVrr,

    JMPi,
    JMPr,
    CJMPi,
    CJMPr,
    CALL,
    RET,
    END,

    MLOADi,
    MLOADr,
    MSTOREi,
    MSTOREr,

    RANGECHECK,
    AND,
    OR,
    XOR,
    NOT,
    NEQ,
    GTE,

    PROPHET,

    Phi,
\end{lstlisting}

Operand data type:
\begin{lstlisting}[language={}]
    Reg(Reg),
    VReg(VReg),
    Int8(i8),
    Int32(i32),
    Int64(i64),
    MemStart,
    Slot(SlotId),
    Block(BasicBlockId),
    Label(String),
    GlobalAddress(String),
    None,
\end{lstlisting}

\end{itemize}
\subsubsection{ABI Lower: Lowering Function Call}
    
Ola Procedure Call Standard (OPCS) are as follows:

The stack initialization points to the first address of the frame stack after the \texttt{fp} register is loaded.
    
The address will be increased when the \texttt{call} instruction is executed later.
When the \texttt{ret} instruction is executed, the \texttt{fp} register points to the address and falls back.
    
    
The Calling process is as follows:
\begin{itemize}
    \item call label

Caller use \texttt{call} instruction to call a callee as \texttt{call functionLabel}, and \texttt{fp} points to the new frame.\par
The \texttt{pc} address returned by the callee is placed in \texttt{fp-1} which is detected by VM but not visible by the compiler backend.\par
Its instructions pattern are as follows:
\begin{lstlisting}[language={}]
main:
.LBL0_0:
  ...
  call foo
  ...
foo:
.LBL1_0:
  ...
\end{lstlisting}
    \item function address

The address pointed to by \texttt{fp} before the function call is placed in \texttt{fp-2} as \texttt{mstore [r8,-2] r8}.\par
Its instructions pattern are as follows:
\begin{lstlisting}[language={}]
mstore [r8,-2] r8
\end{lstlisting}
    \item passing arguments

Function parameter processing: the first three input parameters are placed in the three registers \texttt{r1}, \texttt{r2}, and \texttt{r3}.
If there are more than 3 parameters, start with the fourth input parameter and descend accordingly in \texttt{fp-3}, \texttt{fp-4}, \texttt{...}. \par
Its instructions pattern are as follows:
\begin{lstlisting}[language={}]
mov r1 vreg1
mload r2 [r8,offset]
mov r3 vreg2
\end{lstlisting}
    \item  local variables

Local variables inside the function start at old \texttt{fp}, and their addresses are stored incrementally.

The single return value is stored in \texttt{r0}. If there are multi return values, it needs to be returned by a memory pointer that return the package data.\par
Instruction pattern for single return value is as follows:
\begin{lstlisting}[language={}]
mov r0 vreg3
\end{lstlisting}
\end{itemize}

The call stack frames layout is as follows \ref{fig:ola-lang-backend-functioncall}:

\begin{figure}[!htp]
    \centering
    \includegraphics[width=0.8\textwidth]{ola-lang-backend-functioncall.jpg}
    \caption{Ola-lang Function Call Stack Frames}
    \label{fig:ola-lang-backend-functioncall}
\end{figure}

For prophet library functions, its instructions pattern as:
\begin{lstlisting}[language={}]
.PROPHET{funcNum}_{prophetNum}:  // bind to prophet label
mov r0 psp  // interact with prophet read-only memory, get return value from prophet pointer
mload r0 [r0,0]  // used returned r0 as indexed addressing
\end{lstlisting}

First \texttt{.PROPHET} label binds to the prophet instance in assembly output.
Then the program interacts with prophet read-only memory, get the return value from prophet pointer \texttt{[psp]} and write the result into \texttt{r0}.
At last, we use \texttt{r0} as indexed addressing to load return values from prophet memory.

\subsubsection{Insts selection: match pattern from ir inst to mcinst}

It match ir insts opcode + operand pattern, then lower the matched pattern to target machine code insts.

Several common patterns such as: 
\begin{table}[!ht]
    \resizebox{\textwidth}{!}{
        \begin{tabular}{|c|c|}
            \hline
            \textit{Pattern Type} & \textit{Description} \\ \hline
            Alloca  & params and vars allocation \\ \hline
            IntBinary & bianary operator \\ \hline
            Load & memory load \\ \hline
            Store & memory store  \\ \hline
            Call & function call \\ \hline
            Return & function call return \\ \hline
            Branch & branch control flow \\ \hline
            Conditional Branch & conditional branch control flow \\ \hline
        \end{tabular}}
    \caption{Instruction pattern}
    \label{table:Instruction-pattern}
\end{table}

Let's take Conditional Branch for example, it's target insts as follow:
\begin{table}[!ht]
    \resizebox{\textwidth}{!}{
        \begin{tabular}{|c|c|c|c|}
            \hline
            \textit{operator}  & \textit{Reg+Imm} & \textit{Reg+Reg} & \textit{Cycles}  \\ \hline
            == & Mov tmpReg imm; \par Eq  tmpReg regA tmpReg; \par Cjmp regA labelTrue & Eq  tmpReg regA regB;\par Cjmp tmpReg labelTrue & 2inst + 2reg; 1inst + 3reg \\ \hline
            <  & Mov tmpReg imm; lt  tmpReg regA tmpReg; Cjmp regA labelTrue & lt  tmpReg regA regB; Cjmp tmpReg labelTrue & 2inst + 2reg; 1inst + 3reg \\ \hline
            <= &  &   &  \\ \hline
            >  &  &   &  \\ \hline
            >= &  &   &  \\ \hline
            != &  &   &  \\ \hline
        \end{tabular}}
    \caption{Conditional Branch Pattern}
    \label{table:conditional-branch-pattern}
\end{table}
\subsubsection{Slot elimination}

This pass handle the stack slot for local variables.

Its pipeline as follow:
\begin{lstlisting}[language={}]
VistModule
    | VisitFunction layout
        | VisitBasicBlock
            | Match inst's data operand is Slot type
                | workList: push inst 
    | computer slot offset
    | foreach workList
        | fixup inst's operand with offset and size
\end{lstlisting}
\subsubsection{Target Instruction Insertion: Prologue and Epilogue}

When the processing is completed after the parameters of the function and the function body, 
as a part of the function it needs to do the corresponding stack space processing at the entrance and exit, respectively.
That is, first calculate the stack size, then the stack is opened at the entrance, and the stack is recycled at the exit.

When there is no function call in the function body, entrance is one add instruction such as:
\begin{lstlisting}[language=rust]
InstructionData {
    opcode: Opcode::ADDri,
    operands: vec![
        Operand::input_output(GR::R8.into()),
        Operand::input_output(GR::R8.into()),
        Operand::input(adj.into()),
    ],
}
\end{lstlisting}

When there is one or multi function call in the function body, entrance is one add inst and one mstore instruction such as:
\begin{lstlisting}[language=rust]
InstructionData {
    opcode: Opcode::ADDri,
    operands: vec![
        Operand::input_output(GR::R8.into()),
        Operand::input_output(GR::R8.into()),
        Operand::input(adj.into()),
    ],
}

InstructionData {
    opcode: Opcode::MSTOREr,
    operands: vec![
        Operand::output(GR::R8.into()),
        Operand::input(GR::R8.into()),
    ],
}
\end{lstlisting}

While the exit is one sub operator which is expressed as add instruction such as:
\begin{lstlisting}[language=rust]
InstructionData {
    opcode: Opcode::ADDri,
    operands: vec![
        Operand::output(GR::R8.into()),
        Operand::input_output(GR::R8.into()),
        Operand::input((-adj).into()),
    ],
}
\end{lstlisting}

\subsubsection{Register allocation and coalescing}

register allocation use linear scan method:\par
(a) Analyze liveness in function, for input and output find live in and live out.\par
(b) Insert spill and reload code, push to worklist\par
(c) Rewrite virtual register for target register.

register coalescing pile line:

(a) Foreach the movrr MCInst at basicblock of function on module.\par
(b) If two register is same, then push it into worklist.\par
(c) Remove the inst in worklis from function.\par
\subsubsection{Assembly Printing}

Program basic format\par
The basic format of the Ola assembly language is as follows:
\begin{lstlisting}[language={}]
{symbol} {instruction | directive | pseudo-instruction} {; | // comment}
\end{lstlisting}

\begin{itemize}
    \item Symbol indicates a symbol, which must start at the beginning of the line.\par
    \item Instruction indicates an instruction, it is usually preceded by two spaces.\par
    \item Directive indicates a pseudo operation.\par
    \item Pseudo instruction means a pseudo instruction.\par
    \item Directives, pseudo operations, and pseudo instruction helpers are all case-sensitive, but cannot be mixed.
\end{itemize}

Assembly instructions\par

For simplicity, pseudo operations and pseudo instructions like \texttt{.global} is not considered for now.
Function entries that start with \texttt{funcName} and end with \texttt{:} are treated as label. For example, \texttt{main:} defines a label for a function named main.

Note: The symbols that usually start with \texttt{.} symbols that begin with \texttt{.} indicate pseudo directives or pseudo operations, such as different segments.
Symbols ending with \texttt{:} indicate labels, such as function names and basic block numbers.

Instruction Format\par
The format of the internal assembly instruction is in the form of a three-address code:
\begin{lstlisting}[language={}]
    <opcode> <Rd> <Rn> <shifter_operand>
\end{lstlisting}

\begin{itemize}
    \item \texttt{Opcode} indicates the instruction helper, usually the instruction helper defined by OlaVM.\par
    \item \texttt{Rd} indicates the instruction operation destination register, which is usually the register defined by OlaVM.\par
    \item \texttt{Rn} indicates the first source operand of the instruction, usually a register defined by OlaVM.\par
    \item \texttt{shifter\_operand} indicates the instruction data processing operand, usually an immediate or register defined by OlaVM.
\end{itemize}

Memory layout\par
After the program is loaded, \texttt{pc} points to the zero address and the function stack frame is switched according to the hierarchy of function calls,
and the memory address stack grows from a low address to high address.
When prophets are present in program, an indexed addressing register is required to interact with the prophet memory.

