\subsubsection{ISA Description: Register and Instruction}

The register description is as follows:
\begin{table}[!ht]
    \centering
    \begin{tabular}{|c|c|c|}
        \hline
        \textit{Type}  & \textit{Description} & \textit{Register Group}  \\ \hline
        general registers & General used by program & \verb|[r0-r8]| \\ \hline
        return register & Return value for return to caller & \verb|[r0]| \\ \hline
        parameters rigsters & Parameters value for passing arguments & \verb|[r1, r2, r3]| \\ \hline
        temporary registers & Temporary alloc for local variables & \verb|[r4, r5, r6, r7]| \\ \hline
        stack pointer & Function's stack pointer & \verb|[r8]| \\ \hline
        special registers & Interact with vm: pc for program counter and psp for prophet pointer & \verb|[pc, psp]| \\ \hline
    \end{tabular}
    \caption{Register Description}
    \label{table:register-description}
\end{table}

Instruction description contains opcode and operands,
note that the opcode here contains both register and immediate number types in addition to the usual operator.

Opcode with register or immediate number is as follows:
\begin{lstlisting}[language=rust]
    ADDri,
    ADDrr,
    MULri,
    MULrr,

    EQri,
    EQrr,
    ASSERTri,
    ASSERTrr,

    MOVri,
    MOVrr,

    JMPi,
    JMPr,
    CJMPi,
    CJMPr,
    CALL,
    RET,
    END,

    MLOADi,
    MLOADr,
    MSTOREi,
    MSTOREr,

    RANGECHECK,
    AND,
    OR,
    XOR,
    NOT,
    NEQ,
    GTE,

    PROPHET,

    Phi,
\end{lstlisting}

In particular, it is important to note that Operand data types are very different
from the standard LLVM IR such as two register types, memory slot, block and label.

Operand data type in instructions selection is as follows:
\begin{lstlisting}[language=rust]
    Reg(Reg),
    VReg(VReg),
    Int8(i8),
    Int32(i32),
    Int64(i64),
    MemStart,
    Slot(SlotId),
    Block(BasicBlockId),
    Label(String),
    GlobalAddress(String),
    None,
\end{lstlisting}
