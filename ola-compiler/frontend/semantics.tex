
\subsubsection{Semantic Analysis}

The Semantic Analysis phase of the Ola compiler is an extensive process that ensures the program's correctness and consistency. As previously mentioned, this phase consists of several sub-tasks. Here, we delve deeper into each sub-task, providing a more detailed and comprehensive explanation of the process

\subsubsection*{Symbol Resolution}

The compiler analyzes the program's scope and context to resolve symbols accurately. It distinguishes between local and global variables, function declarations, and type definitions. The symbol table, which holds information about each symbol, is updated as the compiler traverses the AST. During this process, the compiler also checks for naming conflicts and multiple declarations, ensuring that the program adheres to Ola's scoping rules.

\subsubsection*{LibFunction Identification}

In the semantic analysis phase, we will identify all libFunction names that users call. We will also construct prototype code for these LibFunctions and verify whether the parameters used to call them match the parameter types and number in the prototype code. If there is a match, we will record them for easy processing of IR generation for Lib Functions in subsequent LLVM IR generation phases.

\subsubsection*{Type Checking}

The compiler ensures that each operation and expression in the program involves operands of compatible types. In this stage, the compiler also infers the types of expressions when necessary and enforces type constraints for function calls, assignments, and arithmetic operations.

\subsubsection*{Control Flow Analysis}

In addition to checking for unreachable code and infinite loops, the control flow analysis process verifies that:

\begin{itemize}
 \item All code paths in a function that should return a value must end with a return statement.
 \item Break and continue statements appear only within loops.
 \item Variables are declared before they are used.
\end{itemize}

\subsubsection*{Constant Expression Evaluation}

During this step, the compiler performs the following tasks:

\begin{itemize}
  \item Evaluates arithmetic and bitwise operations on constant expressions at compile-time, ensuring that the generated code is more efficient.

  \item Detects potential errors, such as undefined behavior or array index out-of-bounds, by evaluating expressions that involve constants.

  \item Folds constant expressions, such as mathematical operations or string concatenations, reducing the code size and improving execution efficiency.

\end{itemize}

\subsubsection*{Semantic Validation}
 The final step of the semantic analysis phase consists of several validation checks, including:

\begin{itemize}
  \item Verifying that variables are initialized before they are used.
  \item Ensuring that variables, functions, and types are declared only once within a given scope.
  \item Checking that all required function arguments are provided and that excess arguments are not supplied.
  \item Validating that return statements are used correctly within functions.

\end{itemize}

The Semantic Analysis phase is crucial for the robustness and correctness of the Ola compiler. By performing these comprehensive checks, the compiler can guarantee that the generated code adheres to the language's semantic rules and is free from errors that might lead to unexpected behavior during execution. With a semantically verified AST, the Ola compiler proceeds to the subsequent phases of the compilation process, ensuring the efficient translation of the source code into executable code tailored for the Ola VM.