\subsubsection{LLVM IR Generation}

The LLVM IR Generation phase is a crucial step in the Ola compiler, as it translates the Abstract Syntax Tree (AST) obtained from the semantic analysis into LLVM Intermediate Representation (IR). This phase leverages the Inkwell framework, a powerful and user-friendly library that simplifies the process of generating LLVM IR code in Rust.

\subsubsection*{Inkwell Initialization}

Initialize the Inkwell context by creating a new \texttt{Context} object. Set up the target information by querying the target machine properties, such as target triple, data layout, and target-specific optimization levels. This information helps in generating the correct LLVM IR code tailored for the target architecture.

The pseudocode is shown below:

\begin{lstlisting}[language=Rust]
// Create a new LLVM context
let context = Context::create();
let module = context.create_module(name);
let builder = context.create_builder();
\end{lstlisting}
\subsubsection*{Lib IR Generation}

Before generating LLVM IR for user-defined functions, the compiler will first generate LLVM IR for previously called lib functions, prophet functions, and builtin functions. These functions have fixed logic that Ola contract developers will not change.

The pseudocode is shown below:

\begin{lstlisting}[language=Rust]
declare_builtins(bin)
declare_prophets(bin)

for each lib function in ns.called_lib_functions:
    if lib function == "u32_sqrt":
        define function u32_sqrt(value: i64) -> i64:
            root = prophet_u32_sqrt(value)
            builtin_range_check(root)
            root_squared = root * root
            builtin_assert(root_squared, value)
            return root
    else if lib function == "u32_xxx":
        // do nothing
    else:
        // do nothing
\end{lstlisting}
\subsubsection*{AST Traversal}

Traverse the AST using a depth-first approach. Focus on visiting function nodes, as they represent the main building blocks of the Ola program. For each function node encountered, generate its corresponding LLVM IR.

The pseudocode is shown below:

\begin{lstlisting}[language=Rust]
fn traverse_ast(node: &AstNode) {
    if let AstNode::Function(func) = node {
        generate_function(&func);
    }

    for child in node.children() {
        traverse_ast(child);
    }
}
\end{lstlisting}

\subsubsection*{Function Generation}

For each Ola function, create a corresponding LLVM function by invoking the \texttt{add\_function} method on the Inkwell module. Map the Ola function's return type and argument types to their corresponding LLVM types using a custom type mapping function. Add the function arguments with their respective types to the LLVM function using the \texttt{append\_parameter} method.

The pseudocode is shown below:

\begin{lstlisting}[language=Rust]
fn generate_function(func: &OlaFunction) -> LLVMFunction {
    // Map Ola function type to LLVM function type
    let func_type = map_ola_type_to_llvm_type(&func.return_type);
    let llvm_func = context.create_function(func.name, func_type);

    // Add function arguments
    for arg in &func.arguments {
        let llvm_arg_type = map_ola_type_to_llvm_type(&arg.type);
        llvm_func.add_argument(llvm_arg_type, arg.name);
    }

    // Create a builder and position it at the entry block
    let builder = context.create_builder();
    builder.position_at_end(&llvm_func.entry);

    // Process each statement in the function body
    for stmt in &func.body {
        process_statement(stmt, &builder);
    }

    // Return the generated LLVM function
    llvm_func
}
\end{lstlisting}

\subsubsection*{Statement Processing}

Based on the type of statement encountered in the function body, generate the appropriate LLVM IR instructions. For expression statements, process the expression and generate the corresponding LLVM IR code. For variable declaration statements, allocate memory for the variable on the stack using the \texttt{build\_alloca} method, and store its initial value using the \texttt{build\_store} method. For control flow constructs, such as loops and conditionals, generate appropriate branching instructions using methods like \texttt{build\_conditional\_branch} and \texttt{build\_loop}.

The pseudocode is shown below:

\begin{lstlisting}[language=Rust]
fn process_statement(stmt: &Statement, builder: &Builder) {
    match stmt {
        Statement::Expression(expr) => process_expression(expr, builder),
        Statement::VariableDeclaration(var_decl) => process_variable_declaration(var_decl, builder),
        // ... other statement types
    }
}
\end{lstlisting}

\subsubsection*{Expression Processing}

Process each expression encountered in the function body and generate the corresponding LLVM IR instructions. This includes handling arithmetic operations, logical operations, and control flow constructs (such as if-else expressions). For each type of expression, call the appropriate function to generate the LLVM IR code. For example, for binary expressions, generate the appropriate arithmetic or logical operation using methods like \texttt{build\_add}, \texttt{build\_mul}, or \texttt{build\_and}. For function calls, generate the \texttt{call} instruction using the \texttt{build\_call} method.


The pseudocode is shown below:

\begin{lstlisting}[language=Rust]
fn process_expression(expr: &Expression, builder: &Builder) -> LLVMValue {
    match expr {
        Expression::BinaryExpression(op, lhs, rhs) => generate_binary_expression(op, lhs, rhs, builder),
        Expression::IfExpression(cond, then_expr, else_expr) => generate_if_expression(cond, then_expr, else_expr, builder),
        // ... other expressions
    }
}
\end{lstlisting}

\subsubsection*{Type Mapping}

Type Mapping is a crucial function that needs to be implemented in order to convert Ola types into their corresponding LLVM types. This mapping function should handle all Ola types, including basic (such as u32, u64, and u256) and complex types (such as structs and arrays). It's important to note that the Ola language is based on the Field type at its core, with various integer types built upon it. The Field Order is \texttt{0xFFFFFFFF00000001}. To fully represent a field element, we create an integer type using the method \texttt{context.i64\_type()}. For complex types such as structs and arrays, custom LLVM types can be created using methods like \texttt{context.struct\_type()} and \texttt{context.array\_type()}.

The pseudocode is shown below:

\begin{lstlisting}[language=Rust]
fn map_ola_type_to_llvm_type(ola_type: &OlaType) -> LLVMType {
    match ola_type {
        OlaType::Field => context.i64_type(),
        OlaType::U32 => context.i32_type(),
        OlaType::U64 => context.i64_type(),
        OlaType::U256 => context.array_type(context.i64_type(), 4),
        OlaType::Struct(name) => context.struct_type(&structs[name].fields, false),
        OlaType::Array(elem_type, size) => context.array_type(map_ola_type_to_llvm_type(elem_type), *size),
        // ... other types
    }
}
\end{lstlisting}
