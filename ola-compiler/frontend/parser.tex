
\subsubsection{Ola Language Parser}
\subsubsection*{Lexical Analysis}
Lexical analysis is the first stage of the compiler frontend. In this phase, the goal is to break down the source code into a series of tokens. The Ola language lexer will handle the following elements:
\begin{itemize}
\item Keywords
\item Identifiers
\item Operators
\item Literals (such as strings, numbers, and boolean values)
\item Comments
\item Delimiters (such as parentheses and commas)
\end{itemize}
Additionally, the lexer will eliminate whitespace and comments, ensuring a clean token stream for the next stage.

\subsubsection*{Syntax Parsing}
Syntax parsing is the process of transforming the tokens generated in the lexical analysis phase into an Abstract Syntax Tree (AST). The Ola language compiler will implement a top-down parser, such as a Recursive Descent Parser, to support Ola language's grammar.

This section will also discuss the implementation of error handling and recovery mechanisms, ensuring that the parser can handle syntax errors gracefully and provide helpful error messages to the user.

\subsubsection*{Abstract Syntax Tree (AST) Generation}
During the syntax parsing phase, the parser will generate an AST representing the program's structure. This section will explain the design of the AST data structures and the process of constructing the AST during parsing. Additionally, it will cover the benefits of using an AST, such as enabling easier manipulation and analysis of the code's structure.

The Ola compiler seamlessly integrates the Lexical Analysis, Syntax Parsing, and Abstract Syntax Tree (AST) Generation processes, forming a cohesive and efficient pipeline. By leveraging the LALRPOP framework, these stages work in harmony, transforming the Ola source code into an AST representation that is suitable for subsequent compiler phases. This unified approach not only simplifies the implementation but also enhances the performance and robustness of the Ola compiler.

By following these steps, the compiler can efficiently convert the Ola source code text into a sequence of tokens:

\begin{itemize}

  \item The first step in implementing the lexical analysis phase of the Ola compiler is to define lexer rules for various token categories, including keywords, identifiers, literals, and operators. These rules should be based on the provided EBNF grammar rules. We created a file named Ola.lalrpop that describes Ola's EBNF grammar rules.

  \item After defining the lexer rules, the next step is to integrate the lexer with the parser. This can be achieved by using the \texttt{lexer} attribute in LALRPOP grammar rules. The \texttt{lexer} attribute specifies which lexer rule should be used to recognize a particular grammar production.

  \item Ola compiler provides error handling and reporting. If the lexer encounters an unexpected character or a malformed token, it generates an error with the corresponding position in the input text. This information can be used to provide helpful error messages to the user.

\end{itemize}

Once the lexical analysis phase is complete, the generated sequence of tokens can be passed to the parser, which will construct an abstract syntax tree (AST) based on the defined grammar rules. This AST can then be further processed by subsequent phases of the Ola compiler, such as semantic analysis, optimization, and code generation, ultimately producing executable code for the target platform.

By leveraging the powerful LALRPOP framework, the Ola compiler can efficiently perform lexical analysis and provide robust error handling, ensuring that the compiler is user-friendly and capable of handling complex Ola source code.