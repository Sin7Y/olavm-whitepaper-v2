\section{Introduction}\label{sec:introduction}

In the past few years, we have witnessed the rapid development of the blockchain industry; the influx of capital, users, and builders has made the entire blockchain industry reach unprecedented prosperity. However, since blockchain is still a brand new field and the infrastructure is not yet complete, there are also problems behind the prosperity. The more prominent problem is the low TPS of blockchain systems, especially the largest public blockchain Ethereum \cite{website:Ethereum}, which in the peak case \cite{website:Etherscan-chart}, can only process up to 23 transactions per second, far less than traditional centralized systems. 

There are many reasons for the congestion of the Ethereum network, such as the POW \cite{website:POW} consensus algorithm (The Merge upgrade \cite{website:The-Merge} to POS \cite{website:POS} consensus now), the repeated execution of verification process of all transactions by nodes, and the serial execution of transactions; for the above reasons, many projects try to scale Ethereum to obtain higher throughput from different dimensions. Approach 1: New public blockchain, which is independent of Ethereum, using faster consensus algorithm or faster transaction execution to get higher TPS, such as BSC \cite{website:BSC}, Solana \cite{website:Solana}, Aptos \cite{website:Aptos}, etc; Approach 2: Ethereum Sidechain, using faster consensus algorithm and then regularly synchronizing to Ethereum, such as Polygon \cite{website:Polygon}, Optimism \cite{website:Optimism}, Arbitrum \cite{website:Arbitrum}, etc; Approach 3: ZK(E)VM, a Layer2 network of Ethereum, using ZK technology to solve the problem of repeated execution of transactions, such as Polygon Hermez \cite{website:Polygon-Hermez}, Zksync \cite{website:Zksync} and so on.

Obviously, in the long term, scaling is not the only problem that has to be solved in Ethereum even the whole blockchain. As there are many excellent teams working hard for scaling, we believe that the next important feature that should be implemented is privacy. Since the blockchain is a public ledger, all transactions that take place on chain are open and transparent, and the asset information of any address is also open and transparent. Excessive information transparency leads to (1)  (Miners / Maximum Extractable Value) problems, miners selectively package transactions according to the fee, resulting in transactions with lower fees has a lower likely hood of being processed, if ever, forcing user to increaset their fees; (2) Huge asset address security problems, in the past year, the total assets stolen by hackers amount to roughly 2 billion dollars; (3) User data ownership problems, both the assets information and transactions information of addresses are open to be monitored and used, which is contrary to the vision of Web3 \cite{website:Web3}. Therefore, when the scaling problem is solved, privacy will become the next urgent feature to be achieved. In the world of blockchain, privacy is not a new topic, it has long been studied and supported by the Zcash team \cite{website:Zcash}.

\subsection{Non Programmable Privacy}

In addition to Zcash, there are other public blockchains facilitating private transactions, such as Monero \cite{website:Monero}, Dash \cite{website:Dash}, Grin \cite{website:Grin}, etc. However, these blockchains only support basic private asset transfers but lack programmability. As a result, their ecosystem development significantly trails Ethereum, which has become the largest public blockchain in the ecosystem due to its programmability. Nevertheless, Ethereum does not have privacy features.
To address this issue, some projects have started exploring ways to introduce privacy to Ethereum, such as the \textit{ZK-ZKRollup} application zk.money, developed by the Aztec team. Unfortunately, the zk.money product has been discontinued, primarily because its privacy features were limited to basic single transfer scenarios. Given the current surge in decentralized finance (DeFi) applications, asset transfers represent just one of the simplest financial scenarios. Consequently, the user base is limited, while maintenance costs continue to accumulate.

Therefore, some projects have begun to explore ways to introduce privacy to Ethereum, such as the \textit{ZK-ZKRollup} application \textit{zk.money} \cite{website:zk.money} developed by the \textit{Aztec} \cite{website:Aztec} team. However, the current \textit{zk.money} product has been discontinued, mainly because its privacy features only apply to simple single transfer scenarios. Given the current explosion of DeFi applications, asset transfer is only one of the simplest financial scenarios, and therefore the user base is limited, while the maintenance costs continue to accrue.
\begin{figure}[!ht]
    \centering
    \includegraphics[width=0.4\textwidth]{Example of Non Programmable privacy.jpg}
    \caption{Example of Non Programmable Privacy}
    \label{fig:Example of Non Programmable Privacy}
\end{figure}

\figref{fig:Example of Non Programmable Privacy} shows the simple logic of non-programmable privacy. The value change logic corresponding to the input and output notes in Section \ref{section: sending-notes} are also fixed, generally in the form of ``$A + B = C + D$''. \textit{Manta Network} \cite{website:Manta-network} is a public blockchain that supports user-defined token privacy transfers, and the privacy transaction constraint circuits of all fungible tokens can be used to reuse the above logic.

A ZK-ZKRollup of a single-use case application, is similar to a corresponding ZKRollup of a single-use case application. If you want to use an asset in another application, you must bridge the asset through another protocol, which is generally a poor user experience. Therefore, just as ZKRollups need to transition to ZK(E)VMs, ZK-ZKRollup also needs to transition to ZK-ZKVMs (Appendix \ref{section: solidity-compatibility} explains how to get solidity compatibility).

\subsection{Programmable Privacy}

Programmability The privacy platform has two main features (1)It supports private 
transactions, not only private transactions. Similar to Zcash \cite{website:Zcash}, users can still 
choose the transaction type, public transaction or private transaction independently; 
(2)Support Programmability, you can deploy any smart contract, public contract or 
private contract, depending on the needs of the project party. Compared with Specific 
Privacy, the main difference is the logic of state transition in Note, from specific 
calculation to arbitrary calculation logic. Figure \ref{fig:Difference between Specific Privacy and Programmable Privacy} simply shows the difference 
between the two.
\begin{figure}[!ht]
    \centering
    \includegraphics[width=0.8\textwidth]{Difference between Specific Privacy and Programmable Privacy.jpg}
    \caption{Difference between Specific Privacy and Programmable Privacy}
    \label{fig:Difference between Specific Privacy and Programmable Privacy}
\end{figure}

The current projects focusing on programmable privacy are Aleo \cite{website:Aleo} and Aztec \cite{website:Aztec}. Aleo \cite{website:Aleo} is a 
privacy public chain, from BTC \cite{website:BTC} to Ethereum \cite{website:Ethereum} to Zcash \cite{website:Zcash} to Aleo \cite{website:Aleo}. It makes up for 
the fact that programmability and privacy cannot coexist at the Layer1 level. 
It has reached the testnet stage and supports developers to deploy privacy contracts; 
Aztec \cite{website:Aztec} focuses on doing Layer2 programmable privacy for Ethereum \cite{website:Ethereum} , a project 
called Aztec3 \cite{website:Aztec3}, is still in development.

There are often two ways to achieve programmability, one is to customize Domain Specific Language (DSL), such as Circom \cite{website:Circom} , Pil \cite{website:Pil}, Noir \cite{website:Noir}, etc.; the other is Smart Contract Language (SCL), 
such as Cairo1.0 \cite{website:Cairo1.0}, Solidity \cite{website:Solidity}, Ola lang \cite{website:Ola-lang} and so on. The main difference is that SCL is defined in the Instruction Set Architecture (ISA) on a General Purpose Language (GPL), compared to DSL, 
with higher abstraction, but also more suitable for writing complex business logic; and DSL abstraction is lower, more suitable for some simple computational expression. 
Take Polygon Hermez's \cite{website:Polygon-Hermez} Pil \cite{website:Pil} language as an example, you can directly use it to define a simple micro-op, such as `A * B + C`, or `A * B * C + D` and other simple combinations. 
Table \ref{table:Difference between DSL and SCL} briefly shows some of the differences between DSL and SCL.

\begin{table}[!ht]
    \centering
    \begin{tabular}{|l|l|l|l|l|l|}
    \hline
        \emph{Type} & \emph{Abstraction} & \emph{Process} & \emph{Difficulty} & \emph{Examples} & \emph{Notes} \\ \hline
        DSL & low & program -> arith-ops -> ops gadgets & normal & \makecell{circom \\ noir \\ cairo} & \makecell{1. semantic analysis \\ 2. codeGen optimization} \\
        \makecell{SCL \\ (ISA/VM)} & high & program -> bytecodes -> cpu circuit & hard & \makecell{solidity \\ cairo1.0 \\ ola lang} & \makecell{1. need a compiler \\2. re-use LLVM framework} \\
    \end{tabular}
    \caption{Difference between DSL and SCL}
    \label{table:Difference between DSL and SCL}
\end{table}

For DSL, you may need to predefine some commonly used operators, each corresponding to a circuit, called a Gadget \cite{website:Gadget}; you can use these operators to combine any desired calculation logic and 
encapsulate it into a function, Therefore, each function can be regarded as a specific calculation; but the call and return logic between functions cannot be handled because the op of the function
 call and the corresponding constraints are not handled. For SCL is a general-purpose language defined on top of ISA, not only basic arithmetic instructions, but also call/ret instructions, 
 memory access instructions, etc. Each instruction has corresponding constraints, collectively referred to as Cpu circuit; therefore, no matter how the contract logic is written, after the 
 compilation, it will be converted into bytecodes composed of these instructions, and then constrained by the Cpu circuit.

 Therefore, Ola chose ISA-based to implement programmability, its main considerations:
 \begin{itemize}
 \item ISA-based language has higher abstraction and programmability, allowing developers to write smart contracts with arbitrary logic;
 \item A full-featured zk-friendly VM can be designed to achieve higher system performance;
 \item LLVM-based compilation framework, can be more easily compatible with other high-level programming languages;
\end{itemize}
\subsection{Full-featured ZK-friendly ZKVM}

As mentioned before, the best way to achieve programmability is to design a full-featured zk-friendly ZKVM with a customized instruction set, a customized smart contract language, a customized compiler, etc. 
ZKVM is a virtual machine that can execute any program and at the same time generate a zero-knowledge proof of the correctness of the execution process. Therefore, the speed of proof generation 
is very critical, and it will directly affect the performance of the entire system.

The key to obtaining a full-featured zk-friendly ZKVM is how to obtain (1) the smallest execution trace; (2) the most succinct state transition constraints; (3) the fastest 
zero-knowledge proof algorithm. The smallest execution trace means: for the same computational logic, OlaVM \cite{website:OlaVM} can be expressed with the least instructions, the main technical means are the 
support for non-deterministic computation at the computational level, and the register-based design is used at the memory access level; The most succinct state transition constraints means: 
for the same computational logic, OlaVM \cite{website:OlaVM} can constrain the entire execution trace with the least polynomials and the smallest order. The main means is to obtain the instruction with the least 
number of instructions through the Algebraic RISC architecture. The size of instruction set determines the complexity of CPU circuit; a faster zero-knowledge proof algorithm 
means: for the same calculation, OlaVM \cite{website:OlaVM} can complete the proof generation process in a smaller time, which mainly depends on the Goldilocks \cite{website:Goldilocks} field, a finite field less than 64bit, compared to the 
SNARK system based on large bit width elements of elliptic curves, the STARK algorithm based on the Goldilocks \cite{website:Goldilocks} field can be executed faster.

Subsequent chapters will explain in detail Ola's design philosophy and design specifications for obtaining a full-featured zk-friendly ZKVM. As a programmable privacy Layer2 network 
based on ZKVM, Ola will support the following scenarios for different subjects:

\begin{itemize}
\item For project side
    \begin{itemize}
    \item Developers can freely choose to deploy public contracts (Account-based), privacy contracts (Note-based), and ordinary contracts (Account and Note-based)
        \begin{enumerate}
        \item For public contracts, Ola functions as a ZKVM;
        \item For privacy contracts, Ola functions as a ZK-ZKVM;
        \item For ordinary contracts, Ola functions as a ZK-ZKVM or ZKVM, depending on the user's transaction type;
        \end{enumerate}
    \item Transfer of assets between public and private accounts
    \item Intra-contract, no bridge contract is required, supported by default;
    \item Cross-contract, a bridge contract is required;
    \end{itemize}
\item For user side
    \begin{itemize}
    \item For ordinary contract types, users can freely choose the transaction type;
    \item For public/private contract types, users can only execute transactions of the corresponding type;
    \item Users have a view key to disclose executed private transactions;
    \item Ola supports the update of the view key so that after the view key is exposed, the privacy transactions executed by the user in the future will always be parsed;
    \item Ola supports asset transfers between public and private accounts;
    \end{itemize}
\end{itemize}

\subsection{Outline}

Ola is a full stack developer framework for zero-knowledge applications, the whole framework is shown as figture\ref{fig:Ola framework}:
\begin{figure}[!ht]
    \centering
    \includegraphics[width=0.6\textwidth]{vm/Ola framework.jpg}
    \caption{Ola framework}
    \label{fig:Ola framework}
\end{figure}

The green wires stands for the modules we have implemented, and others stand for the modules we will implement in the future. In the remaining sections of this whitepaper, we will introduce those modules 
as follows:
\begin{itemize}
    \item Section \ref{sec:olavm-a-full-featured-zk-friendly-zkvm} mainly describes the design of the virtual machine of OlaVM, include all tricks to get zk-friendly;
    \item Section \ref{sec:ola-lang} mainly describes the design of customize SM language, Ola-lang and the framework of Ola complier based LLVM;
    \item Section \ref{sec:ola-compiler} mainly describes the design of Ola compiler;
    \item Section \ref{sec:zk-zkvm} mainly describes some key points to get privacy;
    \item Section \ref{sec:algorithms} mainly describes the algorithms used in Ola, include the zk arround and hardware acceleration algorithms;
    \item Section \ref{section:appendix} mainly describes the key features supported in the future and the frameworks;
\end{itemize}
