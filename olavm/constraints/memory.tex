\subsubsection{Constraints for Memory Table} \label{sec:memory-constraints}

\noindent \emph{Regional division of memory}

In general, the memory table is divided into two areas, read-write and write-once, which are indicated in green and yellow respectively in \tabref{table:memory-trace-table}.
In the write-once area, it is subdivided into three small areas according to different uses: prophet area, poseidon area, and ecdsa area, and the available size of each small area is:
$span = 2^{32} - 1$.Define $p =2^{64} - 2^{32} + 1$, $p'=p-span$, $p''=p'-span$, $p'''=p''-span$.The address upper limit for each of these three small areas are $p-1$, $p'-1$, $p''-1$,
The starting addresses of these three small areas are $p'$, $p''$, $p'''$.
Each small area has a corresponding flag column, the correctness of the flag column is ensured by a
range check of the difference between the upper limit and address.Finally, the read-write area is in a four-choice relationship with the three small areas mentioned above, and the
flag of the read-write area can be constrained by this relationship.

\noindent \emph{Constraints}

opcode:
\[\mathrm{opcode}_i \cdot (\mathrm{opcode}_i - op\_mload) \cdot (\mathrm{opcode}_i-op\_mstore) \cdot (\mathrm{opcode}_i-op\_call) \cdot (\mathrm{opcode}_i-op\_ret)=0 \]

When opcode is 0, is\_rw must be 0:
\[ (\mathrm{opcode}_i-op\_mload) \cdot (\mathrm{opcode}_i-op\_mstore) \cdot (\mathrm{opcode}_i-op\_call) \cdot (\mathrm{opcode}_i-op\_ret) \cdot \mathrm{is\_rw}_i=0 \]

The relationship between is\_write and opcode:

For the write case, the opcode may be mstore, call, prophet write;For read, opcode may be mload, call and ret;When opcode is call, it may be read or write, need not constraint for this case:
\[ (\mathrm{opcode}_i-op\_mload) \cdot (\mathrm{opcode}_i - op\_ret) \cdot (\mathrm{opcode}_i - op\_call) \cdot (1-is\_write) = 0 \]
\[ \mathrm{opcode}_i \cdot (\mathrm{opcode}_i - op\_mstore) \cdot (\mathrm{opcode}_i - op\_call) \cdot \mathrm{is\_write}_i=0 \]

filter\_looked\_for\_main:
\[ \mathrm{opcode}_i \cdot (1-\mathrm{filter\_lookup\_for\_main}_i)=0 \]

Constraints for diff\_addr:

diff\_addr is mainly used in the rw segment to perform a range check on the difference of addresses to ensure the correctness of the incremental sorting of addresses.The corresponding auxiliary
column diff\_addr\_inv is the reciprocal of diff\_addr (or 0) and is used to limit diff from unpredictable values to 0/1 to constrain it with rw\_addr\_changed.

The diff\_addr constraint will span the boundary between the two regions, except that the diff\_addr after the span does not need to be range checked.
\[ \mathrm{addr}_{i+1} - \mathrm{addr}_i - \mathrm{diff\_addr}_{i+1} = 0 \]

For rw\_addr\_changed:
\[ \mathrm{rw\_addr\_unchanged}_0 = 0 \]
\[ \mathrm{is\_rw}_i \cdot (1 - \mathrm{rw\_addr\_changed}_i - \mathrm{diff\_addr}_i \cdot \mathrm{diff\_addr\_inv}_i) = 0 \]

For region select:
\[ \mathrm{is\_rw}_i + \mathrm{region\_prophet}_i + \mathrm{region\_poseidon}_i + \mathrm{region\_ecdsa}_i - 1 = 0 \]

Different regions of diff\_addr\_cond have different meanings:
\[ \mathrm{region\_prophet}_i \cdot (p - \mathrm{addr}_i - \mathrm{diff\_addr\_cond}_i) = 0 \]
\[ \mathrm{region\_poseidon}_i \cdot (p'- \mathrm{addr}_i - \mathrm{diff\_addr\_cond}_i) = 0 \]
\[ \mathrm{region\_ecdsa}_i \cdot (p'' - \mathrm{addr}_i - \mathrm{diff\_addr\_cond}_i) = 0 \]
Use region\_prophet, region\_poseidon, and region\_ecdsa as filters to perform range check on diff\_addr\_cond.

Binary constraints:
\[ \mathrm{is\_rw}_i \cdot (1 - \mathrm{is\_rw}_i) = 0 \]
\[ \mathrm{region\_prophet}_i \cdot (1 - \mathrm{region\_prophet}_i) = 0 \]
\[ \mathrm{region\_poseidon}_i \cdot (1 - \mathrm{region\_poseidon}_i)=0 \]
\[ \mathrm{region\_ecdsa}_i \cdot (1 - \mathrm{region\_ecdsa}_i)= 0 \]

Constraints for write-once region:
In the write-once region, the address is incremented by 1 or remains unchanged:
\begin{multline*}
    (1-\mathrm{is\_rw}_i) \cdot (1-\mathrm{region\_prophet}_{i+1}+\mathrm{region\_prophet}_i) \cdot (1-\mathrm{region\_poseidon}_{i+1}+\mathrm{region\_poseidon}_i) \\
    \cdot (\mathrm{addr}_{i+1}-\mathrm{addr}_i) \cdot (\mathrm{addr}_{i+1}-\mathrm{addr}_i-1)=0
\end{multline*}

Write-once constraint, the opcode must be read if the address is not incremented by 1:
\begin{multline*}
    (1-\mathrm{is\_rw}_i) \cdot (1-\mathrm{region\_prophet}_{i+1}+\mathrm{region\_prophet}_i) \cdot (1-\mathrm{region\_poseidon}_{i+1}+\mathrm{region\_poseidon}_i) \\
    \cdot (\mathrm{addr}_{i+1}-\mathrm{addr}_i-1) \cdot \mathrm{is\_write}_{i+1}=0
\end{multline*}

Constraints for read or write: First opcode for each address must be write and the next row must be read and value remains unchanged:
\[ \mathrm{is\_write}_0-1=0 \]
\[ (\mathrm{addr}_{i+1}-\mathrm{addr}_i) \cdot (1-\mathrm{is\_write}_{i+1})=0 \]
\[ (1-\mathrm{is\_write}_{i+1}) \cdot (\mathrm{value}_{i+1}-\mathrm{value}_i)=0 \]

Constraints for rc\_value:
The value of rc\_value in the line where is\_rw is 1 is the non-zero value of diff\_addr and diff\_clk (0 if both are 0); rc\_value equals diff\_addr\_cond when is\_rw is 0.
\[ \mathrm{is\_rw}_i \cdot (\mathrm{diff\_addr}_i + \mathrm{diff\_clk}_i - \mathrm{rc\_value}_i)=0 \]
\[ (1-\mathrm{is\_rw}_i) \cdot (\mathrm{diff\_addr\_cond}_i - \mathrm{rc\_value}_i)=0 \]

For filter\_looking\_rc, it must be 1 in read-write region and must be 1 for read in write-once region.
\[ \mathrm{is\_rw}_i \cdot (1-\mathrm{filter\_looking\_rc}_i)=0 \]
\[ (1-\mathrm{is\_rw}_i) \cdot (1-\mathrm{is\_write}_i) \cdot (1-\mathrm{filter\_looking\_rc}_i)=0 \]
