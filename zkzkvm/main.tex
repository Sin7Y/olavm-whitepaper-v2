\section{ZK-ZKVM} \label{sec:zkzkvm}

OlaVM \ref{section: olavm} is a ZK-VM which is a system that uses zk technology to implement a verifiable circuit system for general computation. However, it has some issues where privacy is required, for example, quotation data of commercial competition, anonymous auction, etc. The ZK-VM's each proof generation leaks the witness data, such as the name and function of the called contract, parameters, and so on.

To address this privacy concern, the ZK-ZK-VM system has been developed. This system builds on top of ZK-VM but adds an extra layer of privacy by ensuring that all witness data is private and does not reveal any information. This is achieved through the use of different permissioned private key systems and a note mechanism similar to ZCash's UTXO model.

The core technology of ZK-ZK-VM is the use of different permissioned private key systems. These keys are used to encrypt and decrypt the witness data, ensuring that it remains private and secure. Additionally, the note mechanism is used to further enhance the privacy of the system. Each note represents a commitment to a certain amount of value, and it is impossible to deduce any information about the transaction from this commitment.

The importance of ZK-ZK-VM lies in its ability to address the privacy concerns that exist in ZK-VM. With ZK-ZK-VM, users can conduct transactions on public blockchains with the assurance that their sensitive information is secure and private. This makes it suitable for a wide range of use cases that require high levels of privacy, such as financial transactions or data sharing.

Furthermore, ZK-ZK-VM also offers the same scalability benefits as ZK-VM. By allowing for off-chain computation and verification, it reduces the burden on the main blockchain and increases transaction throughput. This makes it a more efficient and effective solution for blockchain scaling than traditional solutions.

In Section \ref{section: zkzkvm-framework} we explain our basic framework of ZK-ZKVM. Then, in Section \ref{section: zkzkvm-key}, we explain the key system design in our ZK-ZKVM system. Then, in Section \ref{section: zkzkvm-note}, we explain the note design in our ZK-ZKVM system. Then, in Section \ref{section: zkzkvm-user-end-prove} and \ref{section: zkzkvm-delegable-prove}, we compare two different proof generation schemes.

\subsection{Framework}\label{section: zkzkvm-framework}
\subsection{Key}\label{section: zkzkvm-key}
\subsection{Note}\label{section: zkzkvm-note}
\subsection{User End Prove}\label{section: zkzkvm-user-end-prove}
\subsection{Delegable Prove}\label{section: zkzkvm-delegable-prove}